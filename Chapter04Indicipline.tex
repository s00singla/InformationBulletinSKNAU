{\justifying
	\chapter{INDISCIPLINE}
	\chapterimage{}{5em}
%	\epigraph{The end-product of education should be a man who can walk with logic and live with compassion.}{--- \textup{Dr. S. Radhakrishnan}}
	
	\section*{Rules for Enforcement of Students' Discipline and Good Behavior}
	\section{GENERAL}
	\begin{enumerate}[label=\thesection.\arabic*.]
		\item These rules shall be known as enforcement of students discipline and good behaviour rules.
		\item These rules shall supersede all the previous rules relating to the student's discipline and good behaviour.
		\item These rules shall apply to students of Sri Karan Narendra Agricultural University, Jobner irrespective of place and manner of the act of indiscipline committed by them. It will also include a student enrolled in diploma or certificate course or any other category of course in which instruction/education is imparted by the University.
		\item Indiscipline includes:
		\begin{enumerate}
			\item Continued irregularity in attendance, en-masse absent from classes and negligence in the work assigned.
			\item Causing disturbance or nuisance of any kind, including lockout and gheraos in the classroom, college premises, office, library, hostel, playground, University administrative office \& in any campus of the University as well as other places where the students are officially sent for curricular or extracurricular activities.
			\item Acts of disobedience and defiance of orders, rules and regulations.
			\item Misconduct or misbehaviour or use of unfair means in connection with election of University or student bodies, curricular or extra-curricular activities, functions, examinations and tests of all kinds.
			\item Misconduct or misbehaviour towards a member of the teaching/non-teaching staff of the Institution/University, member of any of the Statutory bodies of the University or any visitor to the University or the Institutions or fellow student(s).
			\item Causing damage to the property of the Institution or the University, disfiguring or abusing the property, including library books and periodicals.
			\item Instigation/Spreading misleading reports or rumours.
			\item Keeping/using/supplying intoxicating drinks or drugs in the College/University campuses, including hostels and playgrounds.
			\item Refusal to produce identity card on demand.
			\item Involvement in any criminal activity or offence during the course of studies inside or outside the campus.
			\item Possession of arms \& weapons in places mentioned in 1.4(b) without prior permission of the Head of the Institution (in case of licensed arms also).
			\item Impersonation on any occasion.
			\item The students found indulged in indisciplinary activities will be liable for inter-college transfer as a part of the punishment. Such transfer will be made by the competent authority of the University on the recommendation of committee and dean of respective colleges.
			\item Any other act in the opinion of disciplinary authority considered to be an act of indiscipline.
		\end{enumerate}
	\end{enumerate}
	
	\section[SUPERVISION AND RESPONSIBILITY]{SUPERVISION OF DISCIPLINE AND SHARING OF RESPONSIBILITY:}
	Discipline shall be supervised at different levels and the responsibility shall be shared by:
	\begin{enumerate}[label=(\alph*)]%[label=\thesection.\arabic*.]
		\item Heads of Institution---Deans/Assoc. Dean/Directors/Assoc. Director/Deputy Director.
		\item Superintendents of Examination Centre, members of flying squad and invigilators.
		\item Director, Students Welfare.
		\item Assistant Director, Students Welfare of College.
		\item Librarian of Central Library.
		\item Assistant Librarian of College Library.
		\item Heads of Hostel Departments.
		\item Chief Warden and Wardens of Hostels.
		\item Director/Asstt. Director/Superintendent Physical Education, Coaches, Tour Incharges, Practical Training Supervisor, Programme Officer-NSS and Commandant-NCC.
		\item Members of teaching faculty and non-teaching staff.
	\end{enumerate}
	\textbf{Notes:}
	\begin{enumerate}[label=(\roman*)]%[label=\thesection.\arabic*.]
		\item Head of Institution means Head of the constituent college/institute of the University and also includes a person discharging duties as such for the time being.
		\item ``Superintendent of Examination Centre'' includes person appointed to act as superintendent, Addl. Supdt., Asstt. Supdt. for University examinations/tests.
	\end{enumerate}
	
	\section{POWERS OF AUTHORITIES:}
	\subsection{Heads of Institutions}
	Heads of Institutions within their jurisdiction shall have the following powers to impose any one or any combination of penalties:
	\begin{enumerate}[label=(\alph*)]
		\item Issue warning.
		\item Impose fine up to Rs. 2,000/-.
		\item Imposition of security deposit which might be confiscated at the discretion in the event of the student being found guilty of indiscipline again.
		\item Placement on conduct probation.
		\item Temporary or permanent withdrawal of concession/aids/stipends/scholarships/fellowships/any other facility.
		\item Debar a student for attending classes up to 15 days.
		\item Permanent or temporary expulsion from hostel.
		\item Deprive a student of library facilities.
		\item Debar a student from participation in games, sports, NCC, NSS and other co-curricular activities.
		\item Disqualify a student from appearing at the next University examination/internal examination including tests.
		\item Expel/rusticate a student up to 2 academic sessions/4 semesters.
	\end{enumerate}
	
	\subsection{Heads/In-Charge of the Department:}
	\begin{enumerate}[label=(\alph*)]
		\item Issue warning.
		\item Impose fine up to Rs. 400/-.
		\item Debar a student from attending classes up to 7 days in the subject/course concerned.
		\item Report cases deserving severe punishment to the Head of the Institution.
	\end{enumerate}
	
	\subsection{ Director Students' Welfare and Assistant Director Students' Welfare:}
	\begin{enumerate}[label=(\alph*)]
		\item Issue warning.
		\item Impose fine up to Rs. 1000/- by DSW and Rs. 200/- by ADSW of the College.
		\item Debar a student from participation in any co-curricular activity for a specified period not exceeding one academic year/two semesters.
		\item Recommend cases deserving severe punishment to the Head of the Institution concerned/the Hon’ble Vice-Chancellor.
	\end{enumerate}
	
	\subsection{Librarian of Central Library/ Assistant Librarians of College libraries shall have powers to:}
	\begin{enumerate}[label=(\alph*)]
		\item Issue warning.
		\item Impose fine up to Rs. 200/-.
		\item Debar a student from the use of library for a period up to two weeks under intimation to the Head of the Institution.
	\end{enumerate}
	\textbf{Note:} Librarian of Central Library means: Honorary Librarian, Deputy Librarian and Librarians.
	
	\subsection{Chief Warden and Wardens of College Hostels:}
	\begin{enumerate}[label=(\alph*)]
		\item Issue warning.
		\item Impose fine up to Rs. 400/- by Chief Warden and Rs. 200/- by the Warden.
		\item Permanent or temporary expulsion of a student from the hostel by the Chief Warden.
		\item Refer cases deserving severe punishment to Head of the Institution through proper channel.
	\end{enumerate}
	
	\subsection{Director/Asstt. Director/ Superintendent Physical Education/ Coaches/ Tour Incharges/ Practical Training Supervisor/ Programme Officer-NSS/ Commandant-NCC:}
	\begin{enumerate}[label=(\alph*)]
		\item Issue warning.
		\item Impose fine up to Rs. 200/-.
		\item Recommend to the Head of the Institution for the removal of a student from the college team/tour/NCC/NSS/training for a specific period.
		\item Report cases deserving severe punishment to the Head of the Institution.
	\end{enumerate}
	
	\subsection{ Member of the teaching faculty:}
	\begin{enumerate}[label=(\alph*)]
		\item Issue warning.
		\item Impose fine up to Rs. 100/-.
		\item Debar a student from his/her class up to 3 days.
		\item Report cases deserving severe punishment immediately with full particulars to the Head of the Department.
	\end{enumerate}
	
	\subsection{ Member of the non-teaching staff:}
	\begin{enumerate}[label=(\alph*)]
		\item Report cases immediately with full particulars to the Head of the Department/Unit Heads.
	\end{enumerate}
	
	\section{CENTRAL DISCIPLINARY COMMITTEE (CDC):}
	\begin{enumerate}[label=\thesection.\arabic*.]
		\item There shall be a Central Disciplinary Committee at the University level, which shall be constituted
		by the Vice Chancellor on a proposal initiated by Director Students' Welfare from time to time. 
		The functions of this committee shall be to enquire into the cases of indiscipline and misbehaviour of
		students where such cases have been referred to the committee by the Dean of the concerned college.
		The committee may call and examine any student, officer, teacher, other employee, or requisition any record
		deemed necessary.
		
		\item After conducting the enquiry, the committee shall forward its report along with advice,
		including the quantum of punishment which in its opinion is proper to be imposed, 
		to the Dean of the concerned college who shall impose the punishment accordingly.
	\end{enumerate}
	
	\section[PROCEDURE FOR PUNISHMENT]{PROCEDURE FOR TAKING COGNIZANCE AND DECIDING ABOUT THE IMPOSITION OF PUNISHMENT/ PENALTIES.}
	\begin{enumerate}[label=\thesection.\arabic*.]
		\item Any employee of the University, any student, or any other person, who has noted any act of indiscipline committed by a student, shall immediately report it to the Dean of the College or Director Students' Welfare.
		\item The Dean of the college concerned and other competent authorities (as mentioned in Rule 2) 
		shall be empowered to enquire into the matter and impose penalties suo moto or on the recommendations 
		of the Standing Disciplinary Committee. Notices of enquiry shall be displayed on college, department,
		and hostel notice boards with a copy to the concerned student(s). Responsibility of obtaining such notices 
		lies with the student.
		\item No penalty of rustication or expulsion shall be imposed unless the student has been given an opportunity to show cause.
		\item The cases of indiscipline may be sent to the Central Disciplinary Committee by the Dean of the
		respective college when other options are exhausted. The CDC may also hold an oral enquiry, ensuring
		reasonable opportunity of defense to the student.
		\item These conditions shall not apply where the order is based on facts leading to a conviction in a criminal court.
		\item Any or all requirements of procedures in 5.2 to 5.4 may be waived by the Dean of the concerned
		college or CDC, if it is not practically possible or if peace and tranquility on campus require immediate action.
		\item The enquiry and punishment procedure may take place \textit{ex-parte} (without prior defense opportunity) under conditions such as:
		\begin{enumerate}[label*=\alph*)]
			\item Delay in proceedings is against the interest of the University.
			\item Notice could not be served due to reasons specified by the competent authority.
			\item The student is unable to join the enquiry for valid reasons.
			\item Expeditious disposal is required to maintain peace on campus.
			\item Adequate circumstantial or direct evidence proves involvement beyond doubt.
		\end{enumerate}
		
		\item Decisions taken under 5.6 and 5.7 by the Dean of the concerned college or CDC shall be final.
	\end{enumerate}
	\noindent\textbf{Note:} There may be a Standing Disciplinary Committee at college level, constituted by the Dean, 
	which shall propose disciplinary action to the Dean based on the enquiry report.
	\section{IMPLICATION OF PUNISHMENT}
	\begin{enumerate}[label=\thesection.\arabic*.]
		\item Any punishment awarded to a student shall be placed in the personal file of the student.
		\item The implication of various punishments shall be as follows
		
		\begin{enumerate}[label=(\alph*)]
			\item \textbf{Warning:} Warning shall be conveyed in writing and placed in the personal file of the student.
			
			\item \textbf{Fine:} Fine shall be imposed in pecuniary terms of the specified amount. Such amount shall be deposited by the student within 7 days of imposition of fine. Failure to deposit such fine will amount to non-fulfillment of the punishment conditions and may lead to striking off the name of the student from the rolls of the University.
			
			\item Imposition of security deposit which might be confiscated at the discretion in the event of the student being found guilty of indiscipline, which will include misdemeanour:
			A specific amount of security in terms of money as per the order will have to be deposited by the student within 7 days of passing the order, it shall be subjected to the condition that if the conduct of the student has been found to be exemplary during the remaining period of his/her stay in the University for which the Dean of the College concerned will give a certificate, the security shall be refunded to him/her. However, in case his/her conduct has been found to benot up to the mark, the security so deposited shall be forfeited. Forfeiture of such security will automatically amount to placing the student on conduct probation for the remaining period ofhis/her stay in the University. In such case, the implication of placement on conduct probation will automatically come into force on such student.
			\item \textbf{Placement on conduct probation:}
			A student, who has been placed on conduct probation, shall be kept under constant watch. The behaviour of such student is expected to be exemplary during the course of conduct probation.He/she is not expected to involve himself/ herself even in any incidence of indiscipline.He/she is expected to be, therefore, more careful in his/her behaviour. In case, he/she commits an act of indiscipline second time again, he/she shall remain in conduct probation for full term of stay and he/she may be rusticated from the University in case of any misconduct during this period, such act shall be considered to be serious. A student so placed on Conduct Probation may be debarred during the period of Conduct Probation to:. 
			\begin{enumerate}[label=\roman*.]
				\item Represent his/her College/University in sports, cultural contests etc., in or outside the
				University.
				\item Hold office in a student's organization, club or society.
				\item Receive any Scholarship, Fellowship or stipend.
				\item Temporary or permanent withdrawal of concession/ aid/ stipends/ scholarships/ fellowships/ any other facility, etc.
				\item The Student for a prescribed period or permanently, as the order may be, shall be debarred to
				avail the facility, which has been withdrawn from him/her by way of punishment.
				\item The student will not be eligible for contesting student’s union election.
			\end{enumerate}
			Permanent or temporary expulsion from Hostel:
			The student shall be denied the facility of hostel for a specific period or permanently as theorder may be. During the period of such punishment in operation, the student will not visit thehostel at all. In case, he/she is found to be visiting the hostel, it shall be considered that thepunishment imposed has not been fulfilled and may lead to striking off the name of the studentfrom the rolls of the University.
			\item The students found indulged in indisciplinary activities will be liable for inter-college transfer
			as a part of punishment.
			\item \textbf{Rustication from the University:}
			Rustication can be as per the orders for a specific period of minimum 2 semesters or one yearas the case may be and a maximum period of 4 semesters or 2 years as the case may beincluding the semester/year in which the act of indiscipline has been committed.
			\item No benefit of any type, including attendance benefit etc. shall be given to a student who has,
			due to the reasons of non-fulfillment of punishment awarded/ invited such inability.
			\item Rustication or expulsion and other various methods be noted in the Character Certificate of the
			student concerned
		\end{enumerate}
	\end{enumerate}
	
	\section{SUSPENSION}
	\begin{enumerate}[label=\thesection.\arabic*.]
		\item A student may be suspended by the Dean if necessary in the interest of the University. Such suspension does not amount to penalty.
		\item Suspension debars a student from availing any University facility.
	\end{enumerate}
	
	\subsection*{Right to Appeal}
	\begin{enumerate}[label=\thesection.\arabic*.]
		\item The student may appeal to the Head of Institution against staff orders within 5 days.  
		\item The student may appeal to the Vice-Chancellor against Head of Institution’s orders within 10 days of issue.  
	\end{enumerate}
	
	\section*{Miscellaneous}
	\begin{enumerate}[label=\thesection.\arabic*.]
		\item Suspended/expelled/rusticated students shall not be admitted to another college/unit without approval of the competent authority.  
		\item All cases of expulsion/rustication shall be reported to the BOM and communicated to all SAUs by the Registrar.  
		\item Examination Superintendents may impose penalties under rules approved by the Academic Council (Resolution No. 7, dated 30-31 August 1990).  
		\item Any disciplinary matter not covered in the above rules shall be dealt with by the Head of Institution.  
	\end{enumerate}
}\clearpage
