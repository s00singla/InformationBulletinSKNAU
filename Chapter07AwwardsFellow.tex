{\justifying
	\chapter{AWARDS, SCHOLARSHIPS \& FELLOWSHIPS}
	\chapterimage{}{5em}
%	\epigraph{If I am successful in........................................................................ unquote}{--- \textup{MS Randhawa}}
	
	\section{AWARDS:}
	
	The following gold medals shall be awarded each year to outstanding candidates on the results of the University examination.
	
	\begin{enumerate}[label=\thesection.\arabic*]
		\item One gold medal will be awarded in each Bachelor’s degree programme every academic year.
		
		\textbf{Eligibility for Awarding Gold Medal at Bachelors Degree Level:}
		\begin{enumerate}[label=(\Alph*)]
			\item A candidate who is topper in the class (stood first) securing at least 80 per cent marks in the total aggregate of the entire degree programme under semester system of examination. Further he/she should have passed each examination in the first attempt in the consecutive academic session without getting supplementary or passed by grace in any subject or a fail in any year. Reappearance in the examination of the walk-out paper will be counted as second attempt.
			\item In case of a tie the medal shall be awarded to the bracketed candidates securing equal percentage of marks in the aggregate.
			\item The candidate devoting more academic year than the prescribed minimum for a degree programme shall not be eligible for award of Gold Medal.
			\item No Gold Medal will be awarded if number of the candidates is less than twenty-five who have passed the degree programme examinations.
			\item The candidate has abided by all the rules and regulations and has not resorted to any act of indiscipline or punished under use of unfair means during examination or convicted of a criminal offence during the entire degree programme.
			\item In case, the candidate standing first in the order of merit and does not fulfill the above condition, he/she shall forfeit his/her claim to the Gold Medal and the same shall be awarded to the next candidate who is otherwise eligible under the rules.
			\item In the similar UG programme running at different constituent colleges of the university the eligibility will be considered jointly irrespective of the campus.
		\end{enumerate}
		
		\item Two best thesis awards, one each for Master’s and Ph.D Degree shall be provided by the University.
		\item One gold medal will be awarded in each subject in each academic year in Master’s Programme. The number of candidates passing the examination shall not be less than 03 in each subject/discipline.
		\item Three gold medals i.e. Chancellor, Vice-Chancellor and University Gold Medal will be awarded each year among all disciplines in Ph.D Programme.
	\end{enumerate}
	
	The detailed guidelines of Gold Medal and Best Thesis Awards for Master’s and Ph.D. scholars are mentioned in Post Graduate Studies Regulations-2024.
	
	
	\section{SCHOLARSHIPS AND FELLOWSHIPS:}
	
	\subsection{University Merit Scholarships:}
	
	\begin{enumerate}[label=\thesubsection.\arabic*]
		\item The University Merit Scholarship of Rs. 1600/- per month upto IV semester, will be awarded to one student in each subject, standing first in order of merit at the Master’s examination of I semester. The eligibility of merit scholarship to M. Sc. (Ag./Hort.) shall be a minimum OGPA of 8.00 out of 10.00 in the first semester.
		\item Only bonafide residents of state of Rajasthan may apply for scholarships.
	\end{enumerate}
	
	\textbf{Eligibility for University Merit Scholarship}
	\begin{enumerate}[label=\alph*)]
		\item Entire SKNAU students who have secured OGPA 8.00 out of 10.00 in I semester may apply for the Merit Scholarship.
		\item In similar PG programme running at different constituent colleges of the university the eligibility will be considered jointly subjectwise irrespective of the campus.
		\item Students must have completed one semester of full-time course work as regular students. However, students not able to secure the required OGPA in the succeeding semester will not continue to get scholarships, and simultaneously, no other student shall be considered for scholarships as new.
		\item He/she is not receiving any other scholarship/fellowship from any other source.
	\end{enumerate}
	
	\textbf{Selection}
	\begin{enumerate}[label=\alph*)]
		\item A list of all eligible students as per above eligibility shall be prepared for each subject by respective subject HoD's at the campus and forwarded to the Faculty Chairman to prepare a combined list. The prepared list shall be provided to a committee (to be formed by the Dean \& Faculty Chairman) for this purpose to scrutinize all the papers/documents submitted by the students to recommend the names to the Dean \& Faculty Chairman.
		\item Three students who secure highest OGPA/Marks in the list in first semester in fulfilling the criteria mentioned in the point No.1, each group shall be awarded the scholarship on the basis of merit.
		\item Once a student is selected for the award of the scholarship, he/she will have to give an affidavit that in case if he/she accepts other scholarship offered to him/her by any other agency at a later date, he/she will surrender the total amount.
	\end{enumerate}
	
	\textbf{Continuation of Scholarship}
	Once a candidate is selected for the award of the merit scholarship on the basis of his/her OGPA in the first semester, he/she will continue to get scholarship for 24 Months for M.Sc. (Ag./Hort.) on a renewed basis provided.
	\begin{enumerate}[label=\alph*)]
		\item The Student must secure at least 8.0 OGPA out of 10.0 throughout his/her degree programme. The student must clear all the courses in each semester. In case the student fails any course, the scholarship shall be stopped forthwith. Clear promotion to the next semester/class is compulsory.
		\item He/she neither fails in any course nor is detained due to a shortage of attendance.
		\item He/she has not dropped any semester or withdrawn from a semester.
		\item He/she is not involved in any act of indiscipline within and outsides the college campus or in any police cases, etc.
		\item He/she has to register as full time student in the semester i.e. carrying load of 15 Cr. Hrs.
		\item He/she has not discontinued studies due to accepting a job or any other contractual arrangements and has registered for the consecutive semester.
		\item He/she should be registered for full-time research in the IV semester to receive this scholarship.
		\item He/she has been a regular student for 4 consecutive semester in M.Sc. without a gap on any grounds whatsoever.
		\item He/she has submitted the thesis not later than five (5) semesters of the registration in M.Sc. (Ag./Hort.).
		\item He/She is not awarded a research fellowship with a contingency grant from any other source.
	\end{enumerate}
	
	\textbf{Other conditions}
	\begin{enumerate}
		\item If a student left the institution on the mid-session, the scholarship shall be stopped w.e.f. the month in which he/she left.
		\item In case of a tie, the scholarship shall be awarded to a student on the basis of preceding degree and further, oldest in the age from amongst the students securing equal OGPA.
		\item The payment of scholarship/fellowship shall be made directly to the bank account of the concerned student awardees opened by him/her. Student has to submit his/her bank account details alongwith a photocopy of the first page of bank passbook.
		\item No scholarship will be paid if the student changes in initial subject allotted for which he/she awarded Merit Scholarship.
		\item The student has not maintained good conduct during the whole study period.
		\item The student should not take any job/part time job during his/her study.
		\item All the representations, if any given by the eligible students shall be routed through his/her Major Advisor and concerned HoD with the factual position and recommendation.
		\item For payment of this scholarship, necessary budget allotment shall be made in time by the Comptroller Office to the College(s).
		\item For any other issue, the Dean \& Faculty Chairman may be authorized to take an appropriate decision.
	\end{enumerate}
	
	\textbf{Note:} Other scholarships/fellowships/incentives will be governed by the Rules of ICAR/UGC/State government/respective organizations etc.
	
	
	\subsection{Scholarships offered by different departments/organizations:}
	
	\begin{enumerate}[label=\thesubsection.\arabic*]
		\item Postmetric Scholarship to SC/ST/OBC/minorities/PH(SAP): All students of UG as well as PG of the categories are awarded scholarship by Department of Social Justice and Empowerment, Govt, of Rajasthan.
		\item National Talent Scholarship: It is awarded by ICAR to the students other than the resident of Rajasthan who admitted in UG programme through All India Joint Entrance Examination, CUET.
		\item Junior Research Fellowship (JRF) and Senior Research Fellowship (SRF): The scholarship is awarded by ICAR to those students who admitted through ICAR in PG and Ph.D. programme respectively.
		\item Rajeev Gandhi fellowship to Ph.D students of SC/ST/OBC/minorities/PH: It is awarded to Ph.D students of category.
		\item Inspired Fellowship sponsored by DST: It is awarded to Ph.D. students.
		\item Incentives for girl students: Financial incentive is given by Department of Agriculture, Govt, of Rajasthan to the girls studying in B.Sc. (Hons.) Ag/Hort./B.Tech.(Dairy Technology) and etc. M.Sc. (Ag./Hort.) Rs. 25000/- per year, whereas, 40,000/- per year for Ph.D. students.
		\item Incentive for Ph.D Scholar by Govt. of Rajasthan: Financial incentive is given by the Govt. of Rajasthan to the Ph.D. students @ Rs. 20,000/- per year.
	\end{enumerate}
	
	\textbf{Note:} For further information about rules \& regulations of the various scholarships and fellowships, the student should contact the office of the Dean of the college concerned. A candidate is eligible to get only one scholarship/fellowship at a time.
	
	\section{THESIS CONTINGENCY GRANT:}
	Thesis contingency amount is also made available to M.Sc. students @ Rs. 500/- and Ph. D. (Course works) students @ Rs. 800/- with the following eligibility conditions:
	\begin{enumerate}[label=\thesection.\arabic*]
		\item Student who has completed course work with minimum OGPA required for degree and has passed his/her comprehensive examination.
		\item He/she should be registered for full time research in the IV semester in M.Sc.(Ag.) and III to VI semester in Ph.D. programme to avail this contingency.
		\item He/she has been a regular student for 4/6 consecutive semesters without a gap on any grounds whatsoever in M.Sc. (Ag.)/Ph.D. programme respectively.
		\item He/she has submitted the thesis not later than 5 semesters of registration for M.Sc. and not later than 7 semesters of the registration for Ph.D. students.
		\item He/she is not awarded a research fellowship with a contingency grant from any other sources.
	\end{enumerate}
}
\clearpage