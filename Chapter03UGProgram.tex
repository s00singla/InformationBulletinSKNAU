{
%	\justifying
	%	\chapterimage{png/Syllabus.png}{3cm}
	\chapter[UNDERGRADUATE PROGRAMME]{UNDERGRADUATE \\ PROGRAMME}
	\chapterimage{}{5em}
%	\epigraph{True education must correspond to the surrounding circumstances or it is not a healthy growth.}{--- \textup{Mahatma Gandhi}}
	
	{\section*{B.Sc. (Hons.) Horticulture}}
	
	\section{Duration of the Programme}
	Students admitted to the degree programme of the Agriculture/ Horticulture / Dairy Technology Faculty shall complete a fixed programme of study distributed over four academic sessions (years) comprising eight semesters.
	
	\section{Minimum/ Maximum Residential Requirement}
	\begin{itemize}
		\item[a.] Minimum residential requirement: \textbf{08 semesters}
		\item[b.] Maximum period for which a student can remain on the College roll: \textbf{14 semesters}
	\end{itemize}
	\begin{NoteBox}
		If a student does not satisfactorily complete his/her course work (5.00 out of 10.00) within the maximum prescribed period, he/she will no longer be a student of the University, and the respective Dean/Principal of the College will drop him/her from the College roll.
	\end{NoteBox}
	\begin{enumerate}
		\item Admission is incomplete without registration of required courses relevant to the degree programme. Hence, all students admitted must go through registration in person on the notified date and shall attend classes from the date of commencement of classes specified by the concerned college/institute for a particular semester. The attendance shall be counted from the date of commencement of the classes.
		\item The student failing to register for the course in a semester within the time allowed shall be deemed to have discontinued during that semester and his/her name shall be dropped from the rolls of the College.
	\end{enumerate}
	
	
	\section{Definitions}
	\begin{enumerate}[label=\thesection.\arabic*.]
		\item \textbf{Academic Year or Academic Session:} of University Runs from July to June and consists of two semesters.
		\item \textbf{Semester:} is an academic term of Normally 18–20 weeks, including final examinations.
		\item \textbf{Course:} A unit of instruction or segment of subject matter to be covered in a semester.Each course is assigned a number, title and credits.
		\item \textbf{Credit Hour:} also written as `Credits' implies that One theory lecture or two practical sessions / field practical each week in a semester.
		\item \textbf{Grade Point:} Numerical value denoting students' performance in a course. (which is usually the marks obtained divided by 10 and rounded off to one decimal point)
		\item \textbf{Credit Point:} Product of credit hours and grade point.
		\item \textbf{SGPA:} Semester Grade Point Average. It is the weighted average of credit points, credits been the weights.
		\item \textbf{OGPA:} Overall Grade Point Average, is numerical value obtained by student in all courses completed in all semesters for a degree programme. It is expressed as rounded to two decimals:
		\[
		OGPA = \frac{\sum (\text{Grade point} \times \text{Credit hrs})}{\sum (\text{Credit hrs})}
		\]
		\item \textbf{Year:} Academic session consisting of two semesters. 
		\item \textbf{Percentage:} For obtaining equivalent percentage of OGPA under 10 point scale, the OGPA will be multiplied by factor of 10. Academic session consisting of two semesters. 
		\item \textbf{Division:} The division of UG students shall be determined by the OGPA at the end of successful completion on programme which is as follows:
		\begin{center}
			\begin{tabular}{|c|c|}
				\hline
				\textbf{OGPA} & \textbf{Division} \\ \hline
				5.00–5.99 & II Division \\ \hline
				5.99–7.49 & I Division \\ \hline
				7.50 and above & I Division with Distinction \\ \hline
			\end{tabular}
		\end{center}
	\end{enumerate}
	
	\section{ Examinations}
	\begin{enumerate}[label=\thesection.\arabic*.]
		\item The University shall conduct a main theory and/or practical examination at the end of each semester. Each exam is of two hours unless specified otherwise.
		\item Weightage: Courses with both theory and practical (viz. Cr.hrs 1+1,2+1,3+1) —70 marks (theory), 30 marks (practical). Courses with only theory or only practical (viz. Cr.hrs. 1+0, 2+0, 3+0, 4+0,  0+1,0+2) —100 marks.
		\item Grade point of a course = \(\text{Total marks obtained}\div{10}\). \\ Credit point = Grade point  \(\times\) Course credit.
		\item Distribution of marks for practical examinations (as per Notification No. SS/SKN/ 2023/3780, dated 17.05.2023):
		
		\begin{table}[H]
			\centering
			\renewcommand{\arraystretch}{1.1} % Adds breathing room for the rows
			\rowcolors{2}{CollegeNavy!5}{white} % Alternating colors
			\begin{tabularx}{\linewidth}{L c c}
				\toprule
				\rowcolor{schedheader}
				\textbf{\color{white}Component} & 
				\textbf{\color{white}Max. Marks (30)} & 
				\textbf{\color{white}Max. Marks (100)} \\	
				\rowcolor{schedheader}
				& \textbf{\color{white}\small Cr.hrs (1+1, 2+1, 3+1)} & 
				\textbf{\color{white}\small Cr.hrs (0+1, 0+2)} \\
				\midrule
				Major Practical Exercise & 9.0 & 30.0 \\ 
				Minor Practical Exercise & 6.0 & 20.0 \\ 
				Comments/Spotting & 3.0 & 10.0 \\ 
				Viva-Voce & 3.0 & 10.0 \\ 
				Practical Record \& Internal Assessment & 6.0 & 20.0 \\ 
				Attendance & 3.0 & 10.0 \\ 
				\bottomrule
			\end{tabularx}
		\end{table}
		
		Attendance marks are allotted proportionately (e.g., up to 75\% = 0.5/2.0 marks; 95–100\% = 3.0/10.0 marks).
		\item The marks obtained in each subject (theory and practical separately) are converted into grade points on a 10-point scale. The minimum grade point required to pass a course shall be 5.00. The minimum grade point required to pass the course(s) in a semester shall be 5.00.
		\item The SGPA shall be calculated at the end of each semester by dividing the sum of the credit points earned by a student in all the courses during the semester by the sum of credit hours of those courses.
		\item A student shall be promoted to the next semester provided he/she secures a minimum SGPA of 4.50 in a semester and earns a minimum of 50\% of the total credit hours offered in that semester.
		\item A student failing to secure the minimum SGPA of 4.50 in a semester shall be placed on \textbf{academic probation} for the next semester. Such a student must improve his/her performance by earning at least 50\% of the credit hours offered in the subsequent semester with a SGPA of 5.00, failing which his/her name shall be removed from the rolls of the University.
		\item A student shall not be allowed to register in the next semester if he/she fails in more than 50\% of the courses offered in the previous semester. His/her registration will be withheld for one semester, and he/she may be allowed to re-register in the following semester with the junior batch, subject to the maximum residential requirement.
		\item A student absent from the semester-end examination without prior permission shall be treated as having failed in that course. In exceptional cases, the Dean/Principal may consider genuine reasons and allow a re-examination.
		\item The student shall not be allowed to appear in the end semester examination if he/she has not attended a minimum of 75\% of the classes held in theory and practical separately. However, the Dean/Principal may condone the deficiency up to 10\% on genuine medical grounds.
		\item If a student fails to appear in the mid-term examination, no re-examination will be conducted. Such cases shall be treated as zero marks for the missed examination, unless there is prior approval from the competent authority.
	\end{enumerate}
	
	\section{Readmission}
	\begin{enumerate}[label=\thesection.\arabic*.]
		\item A student who discontinues studies with the permission of the Dean/Principal may be readmitted within two years, provided the maximum residential requirement is not exceeded.
		\item If a student discontinues without permission, his/her readmission shall not be considered.
		\item In all cases, readmission shall be at the discretion of the University.
	\end{enumerate}
	
	\section{Evaluation}
	\begin{enumerate}[label=\thesection.\arabic*.]
		\item The evaluation shall be based on mid-term examination, semester-end examination, practicals, records, class attendance and performance.
		\item The mid-term examination shall carry 30\% weightage, and the semester-end examination shall carry 70\% weightage for theory courses.
		\item The evaluation of practical courses shall be as per the approved distribution of marks (refer Table in 1.4.4).
	\end{enumerate}
	
	\section{Results and Division}
	\begin{enumerate}[label=\thesection.\arabic*.]
		\item[1.7.1] The results of the examinations shall be declared by the University.
		\item[1.7.2] Division shall be awarded based on OGPA as given in Section 1.3.10.
		\item[1.7.3] A student failing in any course shall not be eligible for award of division until he/she clears all the prescribed courses.
	\end{enumerate}
	
	\subsection{Grace Marks}
	\begin{enumerate}[label=\thesection.\arabic*.]
		\item Grace marks up to a maximum of 0.5 grade point may be awarded to a student in a semester, provided it helps him/her to pass in one course.
		\item Grace marks shall not be awarded to improve division or OGPA.
	\end{enumerate}
	
	\section{Scrutiny and Re-evaluation}
	\begin{enumerate}[label=\thesection.\arabic*.]
		\item Students may apply for scrutiny of their answer scripts within 10 days of result declaration by paying the prescribed fee.
		\item Re-evaluation shall be permissible only in theory courses, as per University rules.
	\end{enumerate}
	
	\section{Attendance}
	\begin{enumerate}[label=\thesection.\arabic*.]
		\item Every student is required to attend a minimum of 75\% of lectures, practicals, library sessions and tutorials.
		\item Attendance shall be counted from the date of commencement of classes, not from the date of admission/registration.
		\item Absence without prior permission shall be treated as unauthorized and may lead to disciplinary action.
	\end{enumerate}
	\newpage
	{\singlespacing
	\section{V Deans' Committee}
	B.Sc. (Hons.) Horticulture Semester-wise Distribution of Courses
	\vspace{-1em}
	\begin{SemesterSchedule}{Semester I}
		1 & FRS-111 & Fundamentals of Horticulture & 3(2+1) \\
		2 & FRS-112 & Principles of Genetics and Cytogenetics & 3(2+1) \\
		3 & NRM-111 & Fundamental of Soil Science & 3(2+1) \\
		4 & BAS-111 & Elementary Statistics and Computer Application & 3(2+1) \\
		5 & BAS-112 & Economics and Marketing & 3(2+1) \\
		6 & BAS-113 & Elementary Plant Biochemistry & 2(1+1) \\
		7 & BAS-114 & Introductory Crop Physiology & 2(1+1) \\
		8 & BAS-115 & Introductory Microbiology & 2(1+1) \\
		9 & BAS-116 & Communication Skills and Personality Development\# & 2(1+1) \\
		10 & NSS-I/NCC-I & National Service Scheme/National Cadet Corp & 1(0+1) (NC)* \\
		\TotalRow{24(14+10)}
	\end{SemesterSchedule}
	\vspace{-2.5em}
	\begin{SemesterSchedule}{Semester II}
		1 & FRS-121 & Tropical and Subtropical Fruits & 3(2+1) \\
		2 & FRS-122 & Principles of Plant Breeding & 3(2+1) \\
		3 & FRS-123 & Plant Propagation and Nursery Management & 2(1+1) \\
		4 & VEG-121 & Summer Vegetable Crops & 3(2+1) \\
		5 & FLS-121 & Principles of Landscape Architecture & 2(1+1) \\
		6 & PHT-121 & Fundamentals of Food Technology and Nutrition & 2(1+1) \\
		7 & NRM-121 & Soil Fertility and Nutrient Management & 2(1+1) \\
		8 & NRM-122 & Water Management in Horticultural Crops & 2(1+1) \\
		9 & BAS-121 & Growth and Development of Horticultural Crops & 2(1+1) \\
		10 & BAS-122 & Information and Communication Technology\#* & 2(1+1) (NC)* \\
		11 & NSS-II/NCC-II & National Service Scheme/National Cadet Corp & 1(0+1) (NC)* \\
		12 & BAS-126 & Physical and Health Education & 1(0+1) (NC)* \\
		\TotalRow{25(13+12)} 
	\end{SemesterSchedule}
	
	\begin{SemesterSchedule}{Semester III}
		1 & FRS-211 & Temperate Fruit Crops & 2(1+1) \\
		2 & VEG-211 & Winter Vegetable Crops & 2(1+1) \\
		3 & VEG-212 & Precision Farming and Protected Cultivation & 3(2+1) \\
		4 & FLS-211 & Commercial Floriculture & 3(2+1) \\
		5 & PPR-211 & Fundamentals of Plant Pathology & 3(2+1) \\
		6 & PPR-212 & Fundamentals of Entomology & 3(2+1) \\
		7 & BAS-211 & Fundamentals of Extension Education & 2(1+1) \\
		8 & BAS-212 & Elementary Plant Biotechnology & 2(1+1) \\
		9 & NRM-211 & Environmental Studies and Disaster Management\# & 3(2+1) \\
		10 & NSS-III/NCC-III & National Service Scheme/National Cadet Corp & 1(0+1) (NC)* \\
		\TotalRow{24(14+10)} 
	\end{SemesterSchedule}
	
	\begin{SemesterSchedule}{Semester IV}
		1 & FRS-221 & Plantation Crops & 3(2+1) \\
		2 & FRS-222 & Breeding of Fruit and Plantation Crops & 3(2+1) \\
		3 & VEG-221 & Spices and Condiments & 3(2+1) \\
		4 & FLS-221 & Ornamental Horticulture & 3(2+1) \\
		5 & PPR-221 & Insect Pests of Fruit, Plantation, Medicinal \& Aromatic Crops & 3(2+1) \\
		6 & PPR-222 & Diseases of Fruit, Plantation, Medicinal and Aromatic Crops & 3(2+1) \\
		7 & NRM-221 & Soil, Water and Plant Analysis & 2(1+1) \\
		8 & NRM-222 & Farm Power and Machinery & 2(1+1) \\
		9 & NSS-IV/NCC-IV & National Service Scheme/National Cadet Corp & 1(0+1) (NC)* \\
		\TotalRow{23(14+9)} 
	\end{SemesterSchedule}
	\clearpage
	\begin{SemesterSchedule}{Semester V}
		1 & FRS-311 & Orchard and Estate Management & 2(1+1) \\
		2 & FRS-312 & Weed Management in Horticultural Crops & 2(1+1) \\
		3 & VEG-312 & Potato and Tuber Crops & 2(1+1) \\
		4 & FLS-311 & Medicinal and Aromatic Crops & 3(2+1) \\
		5 & PHT-311 & Postharvest Management of Horticultural Crops & 3(2+1) \\
		6 & PPR-311 & Insect Pests of Vegetable, Ornamental and Spice Crops & 3(2+1) \\
		7 & PPR-312 & Diseases of Vegetables, Ornamentals and Spice Crops & 3(2+1) \\
		8 & NRM-311 & Introduction to Major Field Crops & 2(1+1) \\
		9 & NRM-312 & Introductory Agroforestry & 2(1+1) \\
		10 & NRM-313 & Agro-meteorology and Climate Change & 2(1+1) \\
		\TotalRow{24(14+10)} 
	\end{SemesterSchedule}
	
	\begin{SemesterSchedule}{Semester VI}
		1 & FRS-321 & Dry Land Horticulture & 2(1+1) \\
		2 & VEG-321 & Breeding of Vegetable, Tuber and Spice Crops & 3(2+1) \\
		3 & VEG-322 & Seed Production of Vegetable, Tuber and Spice Crops & 3(2+1) \\
		4 & FLS-321 & Breeding and Seed Production of Flower and Ornamental Plants & 3(2+1) \\
		5 & PHT-321 & Processing of Horticultural Crops & 3(1+2) \\
		6 & PPR-321 & Apiculture, Sericulture and Lac Culture & 2(1+1) \\
		7 & PPR-322 & Nematode Pests of Horticultural Crops and their Management & 2(1+1) \\
		8 & BAS-321 & Horti-Business Management & 2(2+0) \\
		9 & BAS-322 & Entrepreneurship Development and Business Management & 2(1+1) \\
		10 & NRM-321 & Organic Farming & 2(1+1) \\
		\TotalRow{24(14+10)} 
	\end{SemesterSchedule}
	\clearpage
	\begin{SemesterSchedule}{Semester VII}
		1 & HWE-411 & STUDENT READY - Placement in Industries & 0+10 \\
		2 & HWE-412 & STUDENT READY - Placement in Villages & 0+10 \\
		\TotalRow{20(0+20)} 
	\end{SemesterSchedule}
	
	\begin{SemesterSchedule}{Semester VIII}
		1 & ELP-421 & Commercial Horticulture & - \\
		2 & ELP-422 & Protective Cultivation of High Value Horticultural Crops & - \\
		3 & ELP-423 & Processing of Fruits and Vegetables for Value Addition & - \\
		4 & ELP-424 & Floriculture and Landscape Architecture & - \\
		5 & ELP-425 & Bio-inputs: Bio-fertilizers and Bio-pesticides & - \\
		6 & ELP-426 & Mass Multiplication of Plant and Molecules through Tissue Culture & - \\
		7 & ELP-427 & Mushroom Culture & - \\
		8 & ELP-428 & Bee Keeping & - \\
		\TotalRow{20(0+20)} 
	\end{SemesterSchedule}
	
	\subsection*{Grand Total}
	\textbf{184 (83+101)} \\
	\noindent\textit{The student undergoing ELP may be allowed to register for a maximum of two courses in which they have failed but completed requisite percentage of attendance.} \\
	\noindent *Non-Credit Course \\
	\noindent \# Common Course
	
	\clearpage
	
	\section{VI Deans' Committee}
	\textit{Semester wise course distribution of Undergraduate Program of B.Sc. (Hons.) Horticulture}
	\textbf{(As per Sixth Deans’ Committee Report of ICAR, New Delhi)}
	\begin{SemesterSchedule5}{PART-I, SEMESTER-I}
		% Standard Rows
		BAS-111    & Deeksharambh (Induction cum Foundation Course) & 2 (0+2)* & ADSW \\
		HORT-111   & Fundamentals of Horticulture & 3 (2+1) & Horticulture \\
		FSC-111    & Plant Propagation and Nursery Management & 3 (1+2) & Fruit Science \\
		FLS-111    & Commercial Production of Flower Crops & 3 (1+2) & Floriculture \\
		AGRON-111  & Farming Based Livelihood Systems & 3 (2+1) & Agronomy \\
		AGENGG-111 & Sprinkler and Micro-irrigation Systems & 2 (1+1) & Ag. Engg \\
		ENG-111    & Communication Skills & 2 (1+1) & English \\
		NCC-111    & National Cadet Corps (NCC) / NSS & 1 (0+1) & NCC/NSS \\
		
		% --- Intermediate Header (Skill Enhancement) ---
		% We use \addlinespace for a gap and a colored row for the sub-header
		\addlinespace
		\rowcolor{CollegeNavy!20} \multicolumn{4}{c}{\textbf{Skill Enhancement Courses (SEC-I)}} \\
		\addlinespace
		
		SEC-111 & Landscape Gardening & 2 (0+2)\# & Floriculture \\ 
		SEC-112 & Production of Vermicompost and Bio-organics & 2 (0+2)\# & Soil Science \\
		SEC-113 & Orchard Floor Management & 2 (0+2)\# & Fruit Science \\
		SEC-114 & Mushroom Cultivation & 2 (0+2)\# & Plant Path \\
		SEC-115 & Apiculture & 2 (0+2)\# & Entomology \\
		
		% --- Footer Row (Total) ---
		\midrule
		\rowcolor{white} % Reset color for footer
		\multicolumn{4}{r}{\textbf{Total Credit Hours: 21 (08+13) + 02*}} \\
	\end{SemesterSchedule5}
	
	% Notes are placed OUTSIDE the environment to avoid table breaks
	\noindent \textbf{Notes:}
	\begin{itemize}\setlength\itemsep{0pt}
		\item *NC: Non-gradial courses
		\item \# Skill Enhancement Courses (any two courses to be opted by the student)
		\item SEC-112 is designed as a new skill enhancement course
	\end{itemize}
	
	\clearpage
	\begin{SemesterSchedule5}{PART-I, SEMESTER-II}
		% --- Core Courses ---
		AGRON-121 & Introduction to Major Field Crops & 3 (2+1) & Agronomy \\
		PSM-121   & Commercial Production of Spices and Plantation Crops & 3 (2+1) & PSMA\footnote[0]{Plantation, Spices, Medicinal and Aromatic} Crops \\
		VSC-121   & Plant Propagation and Nursery Management in Vegetables, Flowers and Medicinal crops & 3 (1+2) & Vegetable Science \\
		EXT-121   & Personality Development & 2 (1+1) & Ag. Extension \\
		ECON-121  & Entrepreneurship Development and Business Management & 3 (2+1) & Ag. Economics \\
		ESDM-121  & Environmental Studies and Disaster Management & 3 (2+1) & Soil Sci., Agron., Ento.\\
		NCC-121/NSS-121 & National Cadet Corps (NCC) / NSS (To be continued) & 1 (0+1) & NCC/NSS \\
		
		% --- Intermediate Header (SEC) ---
		\addlinespace
		\rowcolor{CollegeNavy!20} \multicolumn{4}{c}{\textbf{Skill Enhancement Courses (SEC-II)}} \\
		\addlinespace
		
		SEC-121 & Packing and Packaging of Horticulture Crops & 2 (0+2)\# & Horticulture \\
		SEC-122 & Farm Machinery & 2 (0+2)\# & Ag. Engineering \\
		SEC-123 & Introduction to Forestry & 2 (0+2)\# & Hort., Forestry \\
		SEC-124 & Installation, Operation and Maintenance of Micro-irrigation System & 2 (0+2)\# & Ag. Engineering \\
		
		% --- Footer Row ---
		\midrule
		\rowcolor{white}
		\multicolumn{4}{r}{\textbf{Total Credit Hours: 22 (10+12)}} \\
	\end{SemesterSchedule5}
	
	% --- Notes (Outside Environment) ---
	\noindent \textbf{Notes:}
	\begin{itemize}\setlength\itemsep{0pt}
		\item \# Skill Enhancement Courses (any two courses to be opted by the student)
		\item Internship (For 10 weeks: only for exit option for award of UG-Certificate): 10 (0+10)
	\end{itemize}
	
	% ----------------------
\clearpage
\begin{SemesterSchedule5}{PART-II, SEMESTER-III}
	% --- Core Courses ---
	SOIL-211  & Fundamentals of Soil Science & 3 (2+1) & Soil Science \\
	FSC-211   & Commercial Fruit Production & 4 (3+1) & Fruit Science \\
	VSC-211   & Precision Farming and Protected Cultivation & 3 (2+1) & Veg. Sci./Ag. Engg. \\
	VSC-212   & Seed Production of Vegetable, Tuber and Spice crops & 3 (2+1) & Vegetable Science \\
	PPATH-211 & Disease Management of Horticulture crops & 3 (2+1) & Plant Pathology \\
	PEYP-211  & Physical Education, First Aid, Yoga Practices and Meditation & 2 (0+2) & Physical Education \\
	NEMA-211  & Nematode Management in Horticultural Crops & 2 (1+1) & Nematology \\
	NCC-211/NSS-211 & National Cadet Corps (NCC)/ NSS (To be continued) & 1 (0+1)* & NCC/NSS \\
	
	% --- Intermediate Header (SEC) ---
	\addlinespace
	\rowcolor{CollegeNavy!20} \multicolumn{4}{c}{\textbf{Skill Enhancement Courses (SEC-III)}} \\
	\addlinespace
	
	SEC-211 & Computer Programming and Data Structures & 2 (0+2)\# & Comp. Sci./Ag. Stat. \\
	SEC-212 & Turf and Turf Management & 2 (0+2)\# & Floriculture \\
	SEC-213 & Post-harvest Management of Horticulture Produce & 2 (0+2)\# & P.H. Management \\
	SEC-214 & Roof Top Gardening & 2 (0+2)\# & Vegetable Science \\
	
	% --- Footer Row ---
	\midrule
	\rowcolor{white}
	\multicolumn{4}{r}{\textbf{Total Credit Hours: 23 (12+11)}} \\
\end{SemesterSchedule5}

% --- Notes (Outside Environment) ---
\noindent \textbf{Notes:}
\begin{itemize}\setlength\itemsep{0pt}
	\item *NC: Non-gradial course
	\item \# Skill Enhancement Courses (any one course to be opted by the student)
	\item NEMA-211 (1+1) is a newly added course.
	\item SEC-214 is designed as a new skill enhancement course.
\end{itemize}	
	%===============================
	\clearpage
	\begin{SemesterSchedule5}{PART-II, SEMESTER-III}
		% --- Core Courses ---
		SOIL-211  & Fundamentals of Soil Science & 3 (2+1) & Soil Science \\
		FSC-211   & Commercial Fruit Production & 4 (3+1) & Fruit Science \\
		VSC-211   & Precision Farming and Protected Cultivation & 3 (2+1) & Veg. Sci./Ag. Engg. \\
		VSC-212   & Seed Production of Vegetable, Tuber and Spice crops & 3 (2+1) & Vegetable Science \\
		PPATH-211 & Disease Management of Horticulture crops & 3 (2+1) & Plant Pathology \\
		PEYP-211  & Physical Education, First Aid, Yoga Practices and Meditation & 2 (0+2) & Physical Education \\
		NEMA-211  & Nematode Management in Horticultural Crops & 2 (1+1) & Nematology \\
		NCC-211/NSS-211 & National Cadet Corps (NCC)/ NSS (To be continued) & 1 (0+1)* & NCC/NSS \\
		
		% --- Intermediate Header (SEC) ---
		\addlinespace
		\rowcolor{CollegeNavy!20} \multicolumn{4}{c}{\textbf{Skill Enhancement Courses (SEC-III)}} \\
		\addlinespace
		
		SEC-211 & Computer Programming and Data Structures & 2 (0+2)\# & Comp. Sci./Ag. Stat. \\
		SEC-212 & Turf and Turf Management & 2 (0+2)\# & Floriculture \\
		SEC-213 & Post-harvest Management of Horticulture Produce & 2 (0+2)\# & P.H. Management \\
		SEC-214 & Roof Top Gardening & 2 (0+2)\# & Vegetable Science \\
		
		% --- Footer Row ---
		\midrule
		\rowcolor{white}
		\multicolumn{4}{r}{\textbf{Total Credit Hours: 23 (12+11)}} \\
	\end{SemesterSchedule5}
	
	% --- Notes (Outside Environment) ---
	\noindent \textbf{Notes:}
	\begin{itemize}\setlength\itemsep{0pt}
		\item *NC: Non-gradial course
		\item \# Skill Enhancement Courses (any one course to be opted by the student)
		\item NEMA-211 (1+1) is a newly added course.
		\item SEC-214 is designed as a new skill enhancement course.
	\end{itemize}
	\clearpage
	\begin{SemesterSchedule5}{PART-II, SEMESTER-IV}
		% --- Core Courses ---
		VSC-221   & Commercial Vegetable Production & 4 (3+1) & Vegetable Science \\
		FMPE-221  & Farm Power and Machinery for Horticulture & 3 (2+1) & Farm Power \& Mach. Engg. \\
		AGINF-221 & Agricultural Informatics and Artificial Intelligence & 3 (2+1) & Comp. Sci. \\
		HORT-221  & Urban and Peri Urban Horticulture & 2 (1+1) & Horticulture \\
		ECON-221  & Agriculture Marketing and Trade & 3 (2+1) & Ag. Economics \\
		ENTO-221  & Pest Management of Horticulture crops & 3 (2+1) & Entomology \\
		AGRON-221 & Introductory Agro-meteorology and Climate Change & 2 (1+1) & Agronomy \\
		NCC-221/NSS-221 & National Cadet Corps (NCC)/ NSS (To be continued) & 1 (0+1)* & NCC/NSS \\
		
		% --- Intermediate Header (SEC) ---
		\addlinespace
		\rowcolor{CollegeNavy!20} \multicolumn{4}{c}{\textbf{Skill Enhancement Courses (SEC-IV)}} \\
		\addlinespace
		
		SEC-221 & Nursery Production in Horticulture Crops & 2 (0+2)\# & Vegetable Science \\
		SEC-222 & Seed Production Techniques in Vegetable Crops & 2 (0+2)\# & Vegetable Science \\
		SEC-223 & Post-Harvest Processing and Value Addition of Horticultural Crops & 2 (0+2)\# & P.H. Management \\
		
		% --- Footer Row ---
		\midrule
		\rowcolor{white}
		\multicolumn{4}{r}{\textbf{Total Credit Hours: 23 (13+10)}} \\
	\end{SemesterSchedule5}
	
	% --- Notes (Outside Environment) ---
	\noindent \textbf{Notes:}
	\begin{itemize}\setlength\itemsep{0pt}
		\item *NC: Non-gradial course
		\item \# Skill Enhancement Courses (any one course to be opted by the student)
		\item SEC-223 is designed as a new skill enhancement course
		\item Internship (10 weeks, exit option for UG-Diploma): 10 (0+10)
	\end{itemize}
	
	%===============================
	\begin{SemesterSchedule5}{PART-III, SEMESTER-V}
		% --- Core Courses ---
		GPB-311   & Fundamentals of Plant Breeding & 3 (2+1) & Genetics \& Plt. Br. \\
		PPHYS-311 & Growth and Development of Horticultural Crops & 3 (2+1) & Plant Physiology \\
		SOIL-311  & Soil Fertility and Nutrient Management & 3 (2+1) & Soil Science \\
		PPATH-311 & General Microbiology & 3 (2+1) & Plant Pathology \\
		AGINF-311 & Information and Communication Technology in Horticulture & 3 (1+2) & Computer Science \\
		PPHYS-312 & Introductory Crop Physiology & 2 (1+1) & Plant Physiology \\
		STAT-311  & Basic Statistics and Experimental Designs & 3 (2+1) & Ag. Statistics \\
		ET-311    & Educational Tour & 2 (0+2)* & Dean/DSW \\
		NCC-311/NSS-311 & National Cadet Corps (NCC)/ NSS (To be continued) & 1 (0+1)* & NCC/NSS \\
		
		% --- Footer Row ---
		\midrule
		\rowcolor{white}
		\multicolumn{4}{r}{\textbf{Total Credit Hours: 20 (12+08) + 02*}} \\
	\end{SemesterSchedule5}
	
	% --- Notes (Outside Environment) ---
	\noindent \textbf{Notes:}
	\begin{itemize}\setlength\itemsep{0pt}
		\item *NC: Non-gradial course
	\end{itemize}
	\clearpage

\begin{SemesterSchedule5}{PART-III, SEMESTER-VI}
	% --- Core Courses ---
	FOR-321     & Introductory Agroforestry & 3 (2+1) & Forestry/Agronomy \\
	HORT-321    & Laboratory Techniques for Horticultural crops & 2 (0+2) & Horticulture \\
	BIOCHEM-321 & Principles of Biochemistry & 3 (2+1) & Biochemistry \\
	HORT-322    & Dryland Horticulture & 3 (2+1) & Horticulture \\
	ECON-321    & Economics and Marketing & 3 (2+1) & Ag. Economics \\
	AGRON-321   & Principles and Practices of Natural Farming & 2 (1+1) & Agronomy \\
	HORT-323    & Horticulture Based Integrated Farming System & 3 (2+1) & Horticulture \\
	PHM-321     & Processing and Value Addition of Horticultural Crops & 3 (2+1) & Post-Harvest Mgmt. \\
	NCC-321/NSS-321 & National Cadet Corps (NCC)/ NSS (To be continued) & 1 (0+1)* & NCC/NSS \\
	
	% --- Footer Row ---
	\midrule
	\rowcolor{white}
	\multicolumn{4}{r}{\textbf{Total Credit Hours: 22 (13+09)}} \\
\end{SemesterSchedule5}

% --- Notes (Outside Environment) ---
\noindent \textbf{Notes:}
\begin{itemize}\setlength\itemsep{0pt}
	\item *NC: Non-gradial course
\end{itemize}

\clearpage
\begin{SemesterSchedule5}{PART-IV, SEMESTER-VII (Elective Courses)}
	% --- Elective Courses ---
	FSC-411 & Production Technology of Tropical Fruit Crops & 3 (2+1) & Fruit Science \\
	FSC-412 & Production Technology of Subtropical and Temperate Fruit Crops & 3 (2+1) & Fruit Science \\
	VSC-411 & Production Technology of Warm Season Vegetable Crops & 3 (2+1) & Vegetable Science \\
	VSC-412 & Production Technology of Cool Season Vegetable Crops & 3 (2+1) & Vegetable Science \\
	VSC-413 & Protected Cultivation of Vegetable Crops & 2 (1+1) & Vegetable Science \\
	VSC-414 & Breeding of Vegetable Crops & 3 (2+1) & Vegetable Science \\
	FLS-411 & Principles of Landscape Architecture & 3 (2+1) & Floriculture \& Land. \\
	
	% --- Footer Row ---
	\midrule
	\rowcolor{white}
	\multicolumn{4}{r}{\textbf{Total Credit Hours: 20 (13+07)}} \\
\end{SemesterSchedule5}


\noindent Students must opt for \textbf{20 credit hours of electives}. Above are initially suggested electives. A list of additional elective courses is given below.

\subsection*{Additional Electives (offered as per faculty/resources)}
	\begin{itemize}
		\item FSC-413 Breeding of Fruit Crops (3 (2+1))  
		\item FSC-414 Canopy Management in Fruit Crops (3 (2+1))  
		\item FSC-415 Biotechnological Approaches and Micropropagation in Fruit Crops (3 (2+1))  
		\item FSC-416 Production Technology of Arid Fruit Crops (3 (2+1))  
		\item FSC-417 Postharvest Management for Fruit Crops (2 (1+1))  
		\item VSC-415 Production Technology of Tuber Crops (3 (2+1))  
		\item VSC-416 Biotechnological Approaches in Vegetable Crops (3 (2+1))  
		\item VSC-417 Postharvest Management of Vegetable Crops (3 (2+1))  
		\item FLS-412 Turf Management (2 (1+1))  
		\item FLS-413 Protected Cultivation of Flower Crops (3 (2+1))  
		\item FLS-414 Value Addition in Floriculture (3 (2+1))  
		\item FLS-415 Breeding of Ornamental Crops (3 (2+1))  
		\item FLS-416 Commercial Floriculture and Landscaping (3 (2+1))  
		\item FLS-417 Postharvest Handling of Floriculture Crops (3 (2+1))  
	\end{itemize}
	
	\noindent \textbf{Note:} Colleges may choose electives based on faculty and facilities, but total must not exceed 20 credit hours.
	
	%==============================
	\	
	\begin{SemesterSchedule5}{PART-IV, SEMESTER-VIII}
		% --- Core Course (Student READY) ---
		% First column (Code) is empty as per source
		READY & Student READY (RHWE / Industrial Attachment / Project Work / Internship) & 20 (0+20) & Dean \\
		
		% --- Footer Row ---
		\midrule
		\rowcolor{white}
		\multicolumn{4}{r}{\textbf{Total Credit Hours: 20 (0+20)}} \\
	\end{SemesterSchedule5}
	
	% --- Notes (Outside Environment) ---
	\noindent \textbf{Notes:}
	\begin{itemize}\setlength\itemsep{0pt}
		\item Students must register for 20 credits by choosing two modules of (0+10) each.
		\item One/two day orientation will be organized by READY Incharge.
		\item READY incharge (faculty) gets 05 credits workload per batch (not reflected in student transcript).
		\item \textbf{Online Courses:} 10 credits via MOOCs (NPTEL, mooKIT, Coursera, edX, SWAYAM, Udemy etc.), preferably during 3rd/4th year.
		\item MOOCs will be non-gradial; certificates issued by offering institutes.
	\end{itemize}
}
	%===============================
	\section*{Credits Allocation Scheme}
	\begin{table}[h]
		\resizebox{\textwidth}{!}{%
			\begin{tabular}{|c|c|c|c|c|c|c|c|c|c|}
				\hline
				\textbf{Semester} & \textbf{\begin{tabular}[c]{@{}c@{}}Core\\ (Major+Minor)\end{tabular}} & \textbf{MDC} & \textbf{VAC} & \textbf{AEC} & \textbf{SEC} & \textbf{\begin{tabular}[c]{@{}c@{}}Internship/\\ READY\end{tabular}} & \textbf{Total} & \textbf{Non-gradial} & \textbf{Online} \\
				\hline
				I & 11 & 3(2) & -- & 1(3)+2(4) & 4 & -- & 21 & 2(1) & -- \\
				II & 09 & 3(5) & 3(6) & 1(3)+2(4) & 4 & -- & 22 & -- & 10(12) \\
				III & 18 & -- & -- & 2(8) & 2 & -- & 22 & -- & -- \\
				IV & 14 & 3(9) & 3(10) & -- & 2 & -- & 22 & -- & 10(13) \\
				V & 20 & -- & -- & -- & -- & -- & 20 & 2(11) & -- \\
				VI & 22 & -- & -- & -- & -- & -- & 22 & -- & -- \\
				VII & 20 & -- & -- & -- & -- & -- & 20 & -- & -- \\
				VIII & -- & -- & -- & -- & -- & 20 & 20 & -- & --\\
				\hline
			\end{tabular}%
		}
	\end{table}
	
}\clearpage
