{\justifying
	\chapter{RESERVATION POLICY}
		\chapterimage{}{5em}
%	\epigraph{If I am successful in........................................................................ unquote}{--- \textup{MS Randhawa}}
	
	The Reservation Policy for admission in U.G. \& P.G. Programme for various faculties of the University shall be as follows but subject to be altered without notice according to Govt. Regulations.
	
	\section*{1. RESERVED SEATS:}
	
	\subsection*{1.1 Seats to be filled up on All India Basis:}
	20 per cent of the seats for UG and 30 per cent for PG will be filled up by an All India Entrance Examination, conducted by the ICAR, CUET etc.
	
	\subsection*{1.2 Seats to be filled up on Rajasthan Basis:}
	After deducting the number of seats mentioned in clause 1.1, there will be the following reservations:
	\begin{enumerate}[label=(i)]
		\item 16 \% seats are reserved for natural born (not adopted) candidates belonging to Scheduled Castes as notified in the Presidential Order for the State of Rajasthan. 12 \% seats are reserved for natural bom (not adopted) candidates belonging to Scheduled Tribes as notified in the Presidential Order for the State of Rajasthan.
		\item 21 \% seats are reserved for natural born (not adopted) candidates belonging to Other Backward Classes (OBC) as notified by the Government Order for the State of Rajasthan.
		\item Reservation under EWS category will be done as per prescribed rules of state Govt.
	\end{enumerate}
	
	\subsection*{1.3 Reservation for Girls:}
	Out of the seats reserved for each of the categories mentioned in 1.2 and out of those remaining in General Category, 25 \% seats shall be reserved for girls in each category.
	
	\subsection*{1.4 Reservations under Specially Abled Person (SAP) Categories:}
	3 \% seats are reserved for Specially Abled Person (SAP). Admissions to these seats shall be made provided there is a permanent disability not less than 40 per cent and not of the nature which may hamper the functioning of the candidate in his/her profession. The certificate for permanent physical disability including the percentage, issued by the Medical Board duly constituted by the Central/State Government will only be considered. The seats under this category shall be derived from all categories (Boys and Girls) accordingly as per their merit.
	
	\subsection*{1.5 Any other category announced by the State Government/University.}
	
	\section*{2. MODE OF ADMISSION TO RESERVED SEATS:}
	
	\subsection*{2.1 UG Programme:}
	\begin{enumerate}[label=(i)]
		\item The merit list of the successful candidates shall be drawn category-wise. Starting from the top, the admissions will be granted in a category up to the seats available in that category.
		\item If the merit list of any reserved category remains unexhausted after completing reservation quota of each category, the candidates from the remaining list shall be eligible for consideration for admission, on merit, against general category.
		\item Unfilled reserved seats of any category out of clause 1.1, 1.2 (i), (ii) \& (iii) shall be filled up from the candidates of general category.
	\end{enumerate}
	
	The following illustrations shall make the mode of filling reserved seats clearer:
	
	\textbf{Illustration I:} If there are 8 seats reserved for the candidates of SC and the merit of the first two candidates in this category is such that it is higher than the merit of the last candidate getting admission against the General Category i.e. they are eligible to get admission against the General Category, still they will be given admission against the SC Category only and thus a total of 8 and not 10 SC candidates will get admission, as SC candidates.  
	
	\textbf{Illustration II:} If, however, against a quota of 8 seats reserved for SC candidates, the merit of the first 10 candidates in this category is such that it is higher than the last candidate getting admission against General Category, then the 9th and 10th SC candidates will get admission against General Category Seats.  
	
	\textbf{Illustration III:} If there are only 6 candidates who are successful in reserved category against the quota of 8 seats, then all the 6 (six) will be admitted in this category and the remaining 2 seats shall be transferred to the general quota.  
	
	\subsection*{2.2 PG Programme:}
	\begin{enumerate}[label=(i)]
		\item Merit List of success candidates shall be drawn category wise.
		\item The admission will be granted as per the roaster system notified by the State Government from time to time.
		\item In case the candidate of required eligibility after passing Pre-PG Examination in the subsequent year, wants admission in other specialty he/she has to produce,  
		(a) Cancellation of Registration from the University,  
		(b) Certificate of acceptance of his/her resignation from the head of the Institution, with the application of choice of subject and place before admission.
	\end{enumerate}
}\clearpage