{\justifying
	\chapter{RULES OF ATTENDANCE}
	\chapterimage{}{5em}
%	\epigraph{If I am successful in........................................................................ unquote}{--- \textup{MS Randhawa}}
	
	Following guidelines have been framed to maintain uniformity and proper check of student attendance.
	
	\begin{enumerate}
		\item The attendance of each subject of all the students is uploaded on university website regularly as per the directives of H.E. Governor of Rajasthan and chancellor of the University.
		
		\item A student shall be required to maintain a minimum of 75 per cent attendance from the date of commencement of classes in each course to appear in final semester examination.
		
		\item En-masse absence shall be treated as absent in the attendance record of the students.
		
		\item A student who is in short of attendance in one or more course(s) will be detained from appearing in the final examination of such courses (theory and practical both) only in which he/she is short of attendance.
		
		\item The registration of a student in all programmes shall be cancelled on account of continuous absence of 3 classes in a course of 1 credit, 4 classes in a course of 2 or 3 credits and 5 classes in a course of 4 credits with the condition that advance intimation may be given to the student in case extra classes are arranged. However, if a student who has been admitted in the first semester of a programme fails to attend the classes continuously for a period of ten days from the date of commencement of classes without the permission of the Dean/Principal of the college, his/her registration shall be cancelled.\\
		The students will be provided an option for re-registration in the course/programme within seven days of the cancellation of their registration by paying a fee of Rs. 500/- in semester or if a student fails to avail this option, he/she may seek re-registration within two weeks of the cancellation by paying a fee of Rs. 1000/- in semester system. After that, the student will be charged an additional penalty of Rs.100/- per day for up to 30 days by submitting an application to the Dean with valid reasons. However, fulfilment of attendance requirements will be his/her responsibility.
		
		\item If a student who has been admitted to the I year/I Semester of a programme fails to attend the classes continuously for a period of 30 days in semester system without the permission of the Dean of the college, the name of such a student will be removed from the college rolls and he/she may have to seek re-admission next time as a fresh candidate. No petition is permitted in this case.
		
		\item If a regular student of the college in subsequent semester/year fails to register on scheduled time or fails to attend the class after registration continuously for 30 days in semester system, without the permission of the Dean of the college, the student will be removed from the college rolls and parents will be informed accordingly. A student so removed may apply to the Dean within 15 days of his/her removal for reconsideration for re-registration in the next academic session, giving valid reasons. His removal may be revoked, provided that, his/her advisor is satisfied with the performance of the student and the same is approved by the Dean. The period of removal shall be counted towards the number of semester/academic year though no grade/marks would be awarded for this semester/academic year.
		
		\item Students who are deputed by the College/University authorities to represent the College/ University in approved co-curricular activities e.g. Republic day parades, Education tours, Games and sports etc. at college/District/State/National level, will be given the credit of attendance to the extent of the number of lectures delivered during the period devoted towards the journey and participation in connection with the aforesaid activities. The total periods of such absence from college shall not exceed 8 days in a semester.
		
		\item Three years successful training in NCC/NSS and two years in NSO with a minimum of 75\% attendance in UG programme is compulsory for becoming eligible for the degree.
		
		\item In case the total number of classes held in a particular course in a semester is less than 10 per credit hour, the course will be treated as scratched. Such student(s) will be permitted to opt the scratched course only in the ensuing semester when it is normally offered.
		
		\item The attendance of the students registered for research credits \{30 for M.Sc. and 75 for Ph. D\} will be maintained by his major adviser/co-adviser and submitted every month to the Head and action will be taken as per procedure for other course programme.
		
		\item For the purpose of calculating attendance the date of commencement of the classes as applicable in the semester should be taken as base point.
	\end{enumerate}
}\clearpage