{\justifying
	\chapter{INTRODUCTION}
	\vspace{-3cm}
	% Epigraph styling
	\chapterimage{}{5em}
%	\setlength{\epigraphwidth}{0.9\textwidth}
%	\epigraph{If agriculture goes wrong, nothing else will have a chance to go right in our country.}{--- \textup{MS Swaminathan}}
	% Placeholder for the image. Replace with actual filename.
	% \includegraphics[width=\textwidth,keepaspectratio]{jpg/sknau.jpg}
	\tcbincludegraphics[mygraphics, title={SKN Agriculture Univeristy, Jobner}]{jpg/sknau.jpg}

Sri Karan Narendra Agriculture University, formerly a part of the Swami Keshwanand Rajasthan Agricultural University, Bikaner became a separate entity in September 2013 for central Rajasthan covering fifteen districts namely; Ajmer, Jaipur, Dausa, Tonk, Sikar, Bharatpur, Deeg, Neem Ka Thana, Dudu, Kotputali, Dholpur and Alwar. The University is expected to play a very vital role in the development of Agriculture in the broad sense and provide trained human resources, carry out production-oriented research programmes, adoption and propagate new technologies in the field of Agriculture and allied sciences with a view to improve the general economic conditions of the farmers of the state.
\section{Main Objectives}
The main objectives of the University are:
\begin{MissionBox}[Objectives]
	\begin{enumerate}
		\item To impart education in agriculture and allied branches of study;
		\item To introduce advancement of learning and research work in agricultural and allied sciences;	
		\item To undertake extension education programmes especially for rural people of the state of Rajasthan; 
		\item To undertake such other work, activity or project as the University may deem proper in order to achieve
		the objectives for which it has been established.
	\end{enumerate}
\end{MissionBox}

{\begingroup % Localizes the font change so it doesn't affect the rest of the document
\small      % Reduces font size to fit columns better
\renewcommand{\arraystretch}{1.3} % Row spacing
\rowcolors{2}{CollegeNavy!5}{white} % Alternating colors

% Column Spec: Same as before (L = Auto-wrap X column)
\begin{xltabular}{\linewidth}{@{\hspace{0.5em}} L p{3.9cm} c L @{\hspace{0.5em}}}
	
	% ==========================================
	% 1. HEADER FOR THE FIRST PAGE
	% ==========================================
	\caption{Offices and officers of the University} \\ % Optional Caption
	\toprule
	\rowcolor{schedheader}
	\textbf{\color{white} Office / Unit} & 
	\textbf{\color{white} Name of Officer} & 
	\textbf{\color{white} Contact No.} & 
	\textbf{\color{white} Email `@sknau.ac.in' } \\
	\midrule
	\endfirsthead
	
	% ==========================================
	% 2. HEADER FOR SUBSEQUENT PAGES
	% ==========================================
	\multicolumn{4}{l}{\textit{Continued from previous page}} \\
	\toprule
	\rowcolor{schedheader}
	\textbf{\color{white} Office / Unit} & 
	\textbf{\color{white} Name of Officer} & 
	\textbf{\color{white} Contact No.} & 
	\textbf{\color{white} Email `@sknau.ac.in'} \\
	\midrule
	\endhead
	
	% ==========================================
	% 3. FOOTER FOR BROKEN PAGES
	% ==========================================
	\midrule
	\multicolumn{4}{r}{\textit{Continued on next page...}} \\
	\endfoot
	
	% ==========================================
	% 4. FOOTER FOR THE LAST PAGE
	% ==========================================
	\bottomrule
	\endlastfoot
	
	% ==========================================
	% 5. DATA CONTENT
	% ==========================================
	Kulguru & Prof. Pushpendra Singh Chauhan & 01425-254039 & vc@ \\
	Registrar & Sh. Mukesh Kumar Moond & 01425-254980 & registrar@ \\
	Comptroller & Sh. B. R. Meena & 01425-254988 & comptroller@ \\
	Chairman, Faculty of Agriculture \& Dean, SKNCOA & Dr. M. R. Choudhary & 01425-254022 & dean.skncoa@ \\
	Dean, COA, Lalsot, Dausa & Dr. D. K. Yadav & 7976144702 & dean.coalalsot@ \\
	Dean, COA, Fatehpur-Shekhawati & Dr. Sanjay Attar & 9913147464 & dean.coafatehpur@ \\
	Dean, COA, Kumher, Deeg & Dr. R. K. Meena & 8005842512 & dean.coabharatpur@ \\
	Dean, COA, Navgaon, Alwar & Dr. Suman Khandelwal & 9414774474 & dean.coanavgaon@ \\
	Dean, COA, Baseri, Dholpur & Dr. Juned Akhtar & 9414467446 & dean.coabaseri@ \\
	Dean, COA, Kotputli & Dr. S. S. Manohar & 9414349440 & dean.coakotputli@ \\
	Dean, COA, Kishangarh Bas & Dr. M. P. Yadav & 9460255391 & dean.coakishangarhbas @ \\
	Dean, COA, Jhilai, Niwai-Tonk & Dr. R. S. Meena & 9414543695 & dean.coajhilai@ \\
	Dean, COA, Peethampuri & Dr. S. K. Khandelwal & 9549692964 & dean.coapeethampuri @ \\
	Dean, COA, Bhusawar, Bharatpur & Dr. U. B. Singh & 9414714234 & dean.coabhusawer@ \\
	Dean, SKN College of Agri-Business Management & Dr. M.R. Choudhary & - & dean.sknabm@ \\
	Dean, College of Horticulture, Durgapura & Dr. L. N. Bairwa & 8949898223 & dean.coh@ \\
	Dean, College of Dairy Science \& Tech. & Dr. B. L. Jat & 9414516628 & dean.dairytech@ \\
	Director, RARI, Durgapura & Dr. Harphool Singh & 9414773798 & director.rari@ \\
	Director, Education & Dr. Rakesh Sammauriya & 9829611123 & director.edu@ \\
	Director, Research & Dr. Ummed Singh & 8299837429 & director\_research @ \\
	Director, Extension Education & Dr. R. N. Sharma & 9460643249 & director.ext@ \\
	Director, PME & Dr. N. K. Gupta & 9460987039 & director.pme@ \\
	Director, Students’ Welfare & Dr. K. C. Sharma & 7976584932 & dsw@ \\
	Director, HRD & Dr. A. C. Shivran & 9414517546 & director.hrd@ \\
	Controller of Examination & Dr. Nidhi Kundu & 01425-254981 & coe@ \\
	Coordinator PG Examination & Dr. D.K. Gothwal & 9928845911 & coordinatorpg.exam @ \\
	Estate Officer & Dr. Naveen Kumar & 9468468401 & estateofficer@ \\
	Incharge, CIMCA & Dr. S. K. Sharma & 9414459205 & cimca@ \\
	University Librarian & Dr. L.R. Yadav & 9414518051 & librarian@ \\
	Secretary, Sports Board & Dr. Basant Kr. Bhinchhar & 8052645461 & sports@ \\
	Secretary, Cultural Activities & Dr. D.K. Gothwal & 9928845911 & gothwaldk.pbg@ \\
	Career Development Centre & Dr. K. C. Sharma & 7976584932 & chairman.cdc@ \\
	University Vehicle Pool Officer & Dr. Upendra Singh & 9460515989 & poolofficer@ \\
	
\end{xltabular}
\endgroup
}

Sri Karan Narendra Agriculture University integrates teaching, research, extension education service to fulfill its mandate as the Agricultural University of the state of Rajasthan. The students are in the prime focus of the University as they seek intellectual, personal and cultural development. The University provides high-quality undergraduate and post-graduate education in various fields of agriculture and allied branches.


The University awards the following degrees:
\begin{itemize}
	\item \textbf{Undergraduate}: B.Sc. (Hons.) Ag./Hort. \& B. Tech (Dairy Technology)
	\item \textbf{Postgraduate}: M.Sc. (Ag./ Hort.)
	\item \textbf{Doctorate}: Ph.D.
\end{itemize}

\section{Constituent Colleges and students’ intake:}
Sri Karan Narendra Agriculture University integrates teaching, research, extension education and
service to fulfill its mandate as the Agricultural University of the state of Rajasthan. The students are
in the prime focus of the University as they seek intellectual, personal and cultural development. The
University provides high-quality undergraduate and post-graduate education in various fields of
agriculture and allied branches. There are fourteen constituent colleges of the University and the
intake capacity thereof in Bachelor degree programme is as follows
{\begingroup
	% Use small font to ensure all columns fit on one page width
	\small 
	\setlength{\tabcolsep}{4pt} % Slightly reduce padding between columns
	\rowcolors{2}{CollegeNavy!5}{white}
	\renewcommand{\arraystretch}{1.3} % More vertical space for readability
	
	% Define the column types
	% l = Number
	% X = College Name (wraps text automatically)
	% c = Center alignment for numbers
	\begin{xltabular}{\linewidth}{@{\hspace{.25em}} l X c c c c c @{\hspace{.25em}}}
%		\caption{List of Constituent Colleges and Seat Capacity} \\
		\toprule
		\rowcolor{schedheader}
		\textbf{\color{white} S. No.} & 
		\textbf{\color{white} Name of the College} & 
		\textbf{\color{white} Est.Yr.} & 
		\textbf{\color{white} Normal + EWS} & 
		\textbf{\color{white} Payment} & 
		\textbf{\color{white} ICAR} & 
		\textbf{\color{white} Total} \\
		\midrule
		\endfirsthead
		
		% --- Header for Next Pages ---
		\multicolumn{7}{l}{\textit{Continued from previous page}} \\
		\toprule
		\rowcolor{schedheader}
		\textbf{\color{white} S.No.} & 
		\textbf{\color{white} Name of the College} & 
		\textbf{\color{white} Est.Yr.} & 
		\textbf{\color{white} Normal + EWS} & 
		\textbf{\color{white} Payment} & 
		\textbf{\color{white} ICAR} & 
		\textbf{\color{white} Total} \\
		\midrule
		\endhead
		
		% --- Footer for Intermediate Pages ---
		\midrule
		\multicolumn{7}{r}{\textit{Continued on next page...}} \\
		\endfoot
		
		% --- Footer for Last Page (The Note) ---
		\bottomrule
		\endlastfoot
		
		% --- DATA ---
		1. & S.K.N. College of Agriculture, Jobner & 1947 & 64+6 & 54 & 16 & 140 \\
		2. & College of Agriculture, Lalsot & 2007 & 32+4 & 20 & 8 & 64 \\
		3. & College of Agriculture, Kumher, Deeg & 2013 & 80+8 & 40 & 18 & 146 \\
		4. & College of Agriculture, Fatehpur-Shekhawati, Sikar & 2013 & 80+8 & 40 & 18 & 146 \\
		5. & College of Agriculture, Navgaon, Alwar & 2018 & 40+4 & 20 & 0 & 64 \\
		6. & College of Agriculture, Baseri, Dholpur & 2019 & 40+4 & 20 & 0 & 64 \\
		7. & College of Agriculture, Kotputli & 2019 & 40+4 & 20 & 0 & 64 \\
		8. & College of Agriculture, Kishangarh Bas, Khairthal & 2020 & 40+4 & 20 & 0 & 64 \\
		9. & College of Agriculture, Jhilai, Niwai, Tonk & 2021 & 40+4 & 20 & 0 & 64 \\
		10. & College of Agriculture, Peethampuri, Neem Ka Thana & 2021 & 40+4 & 20 & 0 & 64 \\
		11. & College of Agriculture, Bhusawar, Bharatpur & 2021 & 40+4 & 20 & 0 & 64 \\
		12. & College of Dairy Science \& Technology, Jobner & 2021 & 40+4 & 20 & 0 & 64 \\
		13. & College of Horticulture, Durgapura, Jaipur & 2023 & 40+4 & 20 & 0 & 64 \\
		
		\midrule
		& \textbf{TOTAL} & & \textbf{616+62} & \textbf{334} & \textbf{60} & \textbf{1072} \\
		\midrule
		
		14. & Rajasthan Agricultural Research Institute (RARI), Durgapura, Jaipur1 & 2013 & \multicolumn{4}{c}{-- Only M.Sc. \& Ph. D --} \\
		
	\end{xltabular}
	\endgroup
}
%Old table----
\section{Degree Programmes in Different Colleges/Institutes}
The University awards following degrees at under graduate and post-graduate levels in the courses of Agricultural, Horticulture and Dairy Science

{\begin{table}[h!]
	\centering
	\small % Use small font for dense data
	\renewcommand{\arraystretch}{1.4} % Add breathing room
		% Apply row coloring (starts from row 2)
	\rowcolors{2}{CollegeNavy!5}{white}
		% Column Spec:
	% p{2.5cm} = Programme Name (Fixed width)
	% p{3.5cm} = College Name (Fixed width, slightly ragged right)
	% X = Degree/Disciplines (Auto-expand, wraps text)
	% c = Seats (Center)
	\begin{tabularx}{\linewidth}{p{2.2cm} >{\raggedright\arraybackslash}p{4.5cm} >{\raggedright\arraybackslash} X c}
		\toprule
		\rowcolor{schedheader}
		\textbf{\color{white} Programme} & 
		\textbf{\color{white} College/Institute} & 
		\textbf{\color{white} Degree / Disciplines} & 
		\textbf{\color{white} Seats} \\
		\midrule
		% --- UNDER GRADUATE SECTION ---
		% \cellcolor{white} ensures the stripe doesn't cut through the multirow text
		\multirow{3}{=}{\textbf{Under\\Graduate}} 
		& Constituent Colleges of SKNAU (13) 
		& B.Sc. (Hons.) Ag. / Hort. \& Dairy Tech. (4 Year Degree) 
		& \textbf{1072} \\
		
		& Govt. Colleges affiliated with SKNAU (17) 
		& B.Sc. (Hons.) Ag. 
		& \textbf{1020} \\
		
		& Pvt. Colleges affiliated with SKNAU (10) 
		& B.Sc. (Hons.) Ag. 
		& \textbf{900} \\
		\midrule
		
		% --- POST GRADUATE SECTION ---
		\multirow{3}{=}{\textbf{Post\\Graduate}\\ \textit{[M.Sc. (Ag. / Hort.)]}} 
		& SKN College of Agriculture, Jobner \newline (14 Disciplines)
		& Agronomy, Soil Science, Entomology, Plant Pathology, Genetics \& Plant Breeding, Agril. Economics, Agril. Extension Ed., Plant Physiology, LPM, Agril. Statistics, Bio-Chemistry, Fruit Sci., Veg. Sci., PSM \& Aromatic Crops
		& \parbox{2cm}{\centering $41+17+12$\\$= \mathbf{70}$} \\
		
		& College of Agriculture, Fatehpur (Sikar) \newline (04 Disciplines)
		& Agronomy, Soil Science, Genetics \& Plant Breeding, Plant Pathology, Fruit Sci.
		& \parbox{2cm}{\centering $10+05$\\$= \mathbf{15}$} \\
		
		& RARI, Durgapura \newline (07 Disciplines)
		& Agronomy, Soil Science, Entomology, Plant Pathology, Genetics \& Plant Breeding, Nematology, Vegetable Sci.
		& \parbox{2cm}{\centering $20+13$\\$= \mathbf{33}$} \\
		\midrule
		
		% --- DOCTORATE SECTION ---
		\multirow{3}{=}{\textbf{Doctorate}\\ \textit{(Ph.D.)}} 
		& SKN College of Agriculture, Jobner \newline (09 Disciplines)
		& Agronomy, Soil Science, Entomology, Plant Pathology, Genetics \& Plant Breeding, Agril. Economics, Agril. Extension Ed., Vegetable Sci., PSM \& Aromatic Crops
		& \parbox{2cm}{\centering $16+9+6$\\$= \mathbf{31}$} \\
		
		& College of Agriculture, Fatehpur (Sikar) \newline (01 Discipline)
		& Agronomy
		& \parbox{2cm}{\centering $2+1$\\$= \mathbf{3}$} \\
		
		& RARI, Durgapura \newline (07 Disciplines)
		& Agronomy, Soil Science, Entomology, Plant Pathology, Genetics \& Plant Breeding, Plant Physiology, Vegetable Sci.
		& \parbox{2cm}{\centering $18+13$\\$= \mathbf{31}$} \\
		\bottomrule
	\end{tabularx}
\end{table}
}
\pagebreak

\section{College of Horticulture}

%\begin{figure}[h]
%	\centering
%	% 
%	\begin{tcolorbox}[width=\textwidth, size=tight, halign=center, valign=center, colback=gray!10, rounded corners]
%		\includegraphics[width=.9\textwidth,keepaspectratio]{jpg/CoHPicAI.jpg}
%	\end{tcolorbox}
%\end{figure}

\tcbincludegraphics[mygraphics,title={College Building at RARI Durgapura}]{jpg/CoHPicAIe.jpg}


The college was established in the year 2023 with the purpose of strengthening the state as well as the university in the field of horticulture education. The first batch of the college got enrolled in the academic year 2023-24 in the graduation program B.Sc. (Hons.) Horticulture. The programme follows the semester system and incorporates elements of the National Education Policy (NEP).

The college is situated at Durgapura, Jaipur near the Airport, running in the campus of Rajasthan Agricultural Research Institute (RARI). The intake capacity is 64. The college currently has teaching and non-teaching staff members and boasts modern infrastructure including smart classrooms and laboratories.

\begin{center}
	\includegraphics[width=3cm]{png/titlelogo.png}\\
	\textbf{College Logo}
\end{center}

\begin{MissionBox}[Mission Statement]
	\begin{enumerate}
		\item To impart quality education in horticultural sciences, fostering innovation, scientific thinking, and skill development.
		\item To promote research and technology dissemination for sustainable horticultural production.
		\item To nurture competent professionals with ethical values and leadership qualities.
	\end{enumerate}
\end{MissionBox}

\begin{MissionBox}[Vision Statement]
	To emerge as a leading centre of excellence in horticultural education, research, and innovation—empowering students to become globally competent professionals dedicated to advancing sustainable horticulture.
\end{MissionBox}

%	\clearpage

\subsection*{Faculty at College of Horticulture}

\begin{MissionBox}		
	\begin{wrapfigure}{l}{0.125\textwidth}
		\vspace{-20pt}
		\centering
		\includegraphics[width=0.125\textwidth, height=0.125\textwidth, keepaspectratio]{jpg/Dean.JPG}
	\end{wrapfigure}
	\paragraph{Dr. L.N. Bairwa,}\textbf{Dean, College of Horticulture, Durgapura}.\\
	\small{Land line: 0141-2992057 \\
		M-9414932548, 8949898223, \\
		E-mail: dean.coh@sknau.ac.in}
\end{MissionBox}

	\begin{table}[H]
	\centering
	\resizebox{\textwidth}{!}{%
		\begin{tabular}{@{}ll|ll@{}}
			\toprule
			\multicolumn{1}{c}{\textbf{Name and Details}} &
			\multicolumn{1}{c}{\textbf{Photo}} &
			\multicolumn{1}{c}{\textbf{Name and Details}} &
			\multicolumn{1}{c}{\textbf{Photo}} \\ 
			\midrule				
			\textbf{Dr. H. P. Parewa} &
			\multirow{4}{*}{\includegraphics[width=1.5cm,height=1.7cm]{png/HPParewa.png}} &
			\textbf{Dr. Saurav Singla} &
			\multirow{4}{*}{\includegraphics[width=1.5cm,height=1.7cm]{jpg/photo_placeholder.jpg}} \\				
			Associate Professor &  & Assistant Professor  &  \\
			(Soil Sci. \& Agril. Chem.) &  & (Agril. Statistics) &  \\
			
			Contact: 9468959800 &  & Contact: 9888548577 &  \\
			E-mail: hpparewa.soils@sknau.ac.in &  & ssingla.stats@sknau.ac.in &  \\
			\midrule
			\textbf{Dr. Bharti Shokeen} &
			\multirow{4}{*}{\includegraphics[width=1.5cm,height=1.7cm]{png/Bharti.png}} &
			\textbf{Dr. Ranjna Sirohi} &
			\multirow{4}{*}{\includegraphics[width=1.5cm,height=1.7cm]{png/RanjnaMam.png}} \\
			
			Assistant Professor  &  & Assistant Professor  &  \\
			(English) &  & (Agril. Engineering) &  \\
			Contact: 9999067429 &  & Contact: 8630994784 &  \\
			E-mail: bharti.eng@sknau.ac.in &  &  ranjnasirohi.agengg@sknau.ac.in &  \\
			\midrule
			\textbf{Dr. Ashok Choudhary} &
			\multirow{4}{*}{\includegraphics[width=1.5cm,height=1.7cm]{jpg/Ashok1.jpg}} &
			\textbf{Dr. Vijay Parashar} &
			\multirow{4}{*}{\includegraphics[width=1.5cm,height=1.7cm]{jpg/VijayParashar.jpg}} \\
			
			Assistant Professor (Horticulture) &  & Assistant Librarian &  \\
			Contact: 9079283354 &  & Contact: 9672999560 &  \\
			E-mail: ashok.horti@sknau.ac.in &  & vparashar.library@sknau.ac.in &  \\
			
			\bottomrule
		\end{tabular}%
	}
\end{table}
\vspace{1cm}
\subsection*{Administrative Staff}

\begin{table}[H]
	\resizebox{\textwidth}{!}{%
		\begin{tabular}{@{}ll|ll@{}}
			\toprule
			\multicolumn{1}{l}{\textbf{Name and   Details}} &
			\multicolumn{1}{c}{\textbf{Photo}} &
			\multicolumn{1}{l}{\textbf{Name and   Details}} &
			\multicolumn{1}{c}{\textbf{Photo}} \\ 
			\midrule
			\textbf{Dr. Suresh Chandra   Sharma} &
			\multirow{4}{*}{\includegraphics[width=1.5cm,height=1.7cm]{png/sureshji.png}} &
			\textbf{Mr. Surendra Kumar} &
			\multirow{4}{*}{\includegraphics[width=1.5cm,height=1.7cm]{png/SurendraKumar1.png}} \\
			Assistant Registrar                 &                     & Lab Assistant                     &                     \\
			Contact: 9414370422                 &                     & Contact: 7891007480               &                     \\
			E-mail: scsharma.ar@sknau.ac.in     &                     & E-mail: surendra.la@sknau.ac.in   &                     \\ \midrule
			\textbf{Ms. Ronika Jakhar}          & \multirow{4}{*}{\includegraphics[width=1.5cm,height=1.7cm]{png/ronika.png}} & \textbf{Ms. Pushpa Devi Jat}      & \multirow{4}{*}{\includegraphics[width=1.5cm,height=1.7cm]{png/pushpa.png}} \\
			Lab Assistant                       &                     & Matron                            &                     \\
			Contact: 9928633911                 &                     & Contact: 9461335060               &                     \\
			E-mail: ronika.jakhar@sknau.ac.in   &                     & E-mail: pushpadevi91919@gmail.com &                     \\ \midrule
			\textbf{Ms. Nisha Sharma}           & \multirow{4}{*}{\includegraphics[width=1.5cm,height=1.7cm]{png/nisha.png}} & \textbf{Mrs. Rajkumari Pushpad}   & \multirow{4}{*}{\includegraphics[width=1.5cm,height=1.7cm]{png/rajkumari.png}} \\
			Agriculture Supervisor              &                     & Clerk   Grade-II                  &                     \\
			Contact: 8290717834                 &                     & Contact: 9057548067               &                     \\
			E-mail: nishasharma112003@gmail.com &                     & Email:                            &                     \\ \midrule
			\textbf{Mr. Abhishek Sharma}        & \multirow{4}{*}{\includegraphics[width=1.5cm,height=1.7cm]{png/default.png}} & \textbf{}                         & \multirow{4}{*}{\includegraphics[width=1.5cm,height=1.7cm]{png/default.png}} \\
			Lab Attendant                       &                     &                                   &                     \\
			Contact: 9828659625                 &                     &                                   &                     \\
			E-mail:                             &                     &                                   &                     \\ \bottomrule
		\end{tabular}%
	}
\end{table}
\section*{Academic Activities}
During the session 2023–24, the College of Horticulture, Durgapura, Jaipur organized several academic and co-curricular events for the holistic growth and development of students. These activities emphasized communication skills, confidence building, and participation in diverse events such as presentations, quizzes, essay writing, and poster preparation.

\subsection*{Orientation Programme}
An \textit{Orientation Day} was organized on \textbf{October 11, 2023} for the first batch of B.Sc. (Hons.) Horticulture students. The event was graced by Prof. Balraj Singh, Hon’ble Vice Chancellor, SKNAU Jobner, as the Chief Guest. The Dean, Prof. M.C. Gupta, along with other dignitaries welcomed the new students and introduced them to academic, co-curricular, and examination procedures.

\subsection*{Celebration of Important Days}
The college celebrated several national and thematic days during the year:
\begin{itemize}
	\item \textbf{World Soil Day (Dec 5, 2023):} Theme – “Soil and Water: Source of Life”, emphasizing soil and water conservation.
	\item \textbf{Republic Day (Jan 26, 2024):} Hoisting of the national flag by Dean COH, followed by an address on institutional progress.
	\item \textbf{Dr. B.R. Ambedkar \& Jyotiba Phule Jayanti (Mar 18, 2024):} Joint celebration featuring student presentations and talks by the Hon’ble Vice Chancellor and faculty on social reform and education.
	\item \textbf{“Wisteria 2024” (Mar 21, 2024):} The first fresher’s event, celebrating cultural diversity, talents, and creativity.
	\item \textbf{International Carrot Day (Apr 4, 2024):} Focused on health and nutritional importance of carrots through posters and presentations.
	\item \textbf{Exposure Visit (May 18, 2024):} A study tour to \textbf{SKNAU Jobner}, where students visited academic and research facilities under faculty supervision.
\end{itemize}

% Optional photo placeholder
\begin{figure}[p]
	\centering
	\includegraphics[width=0.9\textwidth]{png/Eventss.png}	   
\end{figure}

\section*{Co-Curricular Activities}

\subsection*{NSS and Cultural Events}
Various NSS-led and cultural initiatives promoted social awareness and environmental responsibility:
\begin{itemize}
	\item \textbf{Swachhta Abhiyaan (Mar 30, 2024):} A campus cleanliness drive involving faculty and students.
	\item \textbf{Voting Awareness Program (Mar 30, 2024):} Focused on youth participation in democratic processes.
	\item \textbf{Say No to Plastic (May 4, 2024):} Created awareness about environmental hazards of plastic use.
	\item \textbf{Finance Awareness Seminar (May 7, 2024):} Conducted by \textbf{Ms. Alka Parikh} from the Mumbai Stock Exchange, focusing on savings and investment options.
\end{itemize}

\subsection*{Sports and Games}
The \textbf{First Annual Sports Meet} was organized on \textbf{October 26, 2023}, featuring eleven track and field events, kabaddi, volleyball, and badminton. Students also represented the college at the \textbf{Inter-College Tournament (Nov 5–9, 2023)} hosted by SKNCOA, Jobner.



\section*{Academic Achievements and Scholarships}
Students excelled in university examinations, with \textbf{Ms. Ritu Kantwa}, \textbf{Ms. Sanyogita Singh}, and \textbf{Ms. Kesar Meena} securing top ranks at the university level.

Out of 62 students, \textbf{38 girl students} received \textit{Incentive Scholarships} from the Department of Agriculture, Government of Rajasthan, and 9 students availed scholarships from the \textit{Department of Social Justice and Empowerment, Jaipur}.
\begin{figure}[p]
	\centering
	\includegraphics[width=0.9\textwidth]{png/Cuttings.png}	   
\end{figure}

\section*{Faculty Participation: Conferences, Training, and Workshops}

The faculty members of the College of Horticulture, Durgapura, have actively engaged in a variety of academic and professional development activities, contributing significantly to research, teaching, and institutional growth. Dr. Bharti Shokeen presented papers and conducted sessions on soft skills and communication at several national and international forums, including AMU, DU, and GGSIPU. Dr. Saurav Singla delivered an invited talk on ``Response Surface Methodology'' at a SERB-DST Workshop organized by the Central University of Haryana and . Dr. Ashok Choudhary took part in the Soil Arthropod Pests Meeting, Millet Conclave, and an ICAR Short Course held at RARI, Durgapura. Dr. Ranjna Sirohi participated in prominent international conferences organized by BRSI, CSIR-NIIST, and MACFAST, where she presented her research on biotechnology and waste valorization. Additionally, all faculty members collectively attended an Orientation Programme organized by the HRD Department of SKNAU, Jobner, during February--March 2024, further enhancing their academic and pedagogical competencies.

	\section*{Research and Publications}

The faculty have made notable contributions to research through publications in reputed national and international journals. Dr. H.P. Parewa has published extensively on topics related to soil fertility, biofortification, and soil health, while Dr. Ranjna Sirohi has contributed research on food technology, bioactive extraction, and sustainability. Dr. Ashok Choudhary’s research focuses on horticultural crop management, pest control, and nano-urea studies, whereas Dr. Saurav Singla has authored papers on statistical modeling and genetic diversity in rice. 

In addition to research articles, faculty members have made substantial contributions to books and book chapters. Dr. Saurav Singla co-authored the book \textit{Democracy, Elections and Good Governance in India} (Rede et al., 2024), while Dr. H.P. Parewa, Dr. R. Sirohi, Dr. Bharti Shokeen, and others contributed chapters to Springer and Elsevier volumes covering soil health, microbial biotechnology, food waste management, and communication studies.


The year 2023–24 marked the \textbf{first full academic session} of the College of Horticulture, Durgapura, distinguished by dynamic academic programs, vibrant student participation, and significant faculty contributions to teaching, research, and extension. The collective efforts of faculty, staff, and students have strengthened the college’s academic foundation and enhanced its visibility within SKNAU, Jobner.

% Optional photo placeholder


\section*{Facilities}

\begin{FeatureBox}{Food Technology Laboratory}
	{\textbf The Food Technology Laboratory} at the College of Horticulture, Durgapura provides an excellent platform for B.Sc. Horticulture students to gain practical knowledge and hands-on experience in food processing, preservation, and quality analysis. The laboratory is equipped with a wide range of instruments and tools that help students understand the scientific principles behind food composition and post-harvest management.
	
	\tcblower % --- RIGHT SIDE: IMAGE ---		
	\centering
	\includegraphics[width=\linewidth, keepaspectratio]{png/LabPic1.png} 
	\captionof{figure}{\small Equipment at Food Technology Laboratory, COH Durgapura}		
\end{FeatureBox}
Key facilities include the Kjeldahl apparatus for protein estimation, Soxhlet apparatus for fat analysis, and a double distillation unit for preparing high-purity distilled water used in experiments. Instruments like the analytical weighing balance, pH meter, and EC meter ensure accuracy in measurement and analysis. The lab also equipped essential equipment such as a mixer grinder, autoclave, cap corking machine, sealing machine, and water bath, which are used for developing and preserving various food products. For microbiological and aseptic work, a laminar air flow unit and microscope are available to ensure safe and sterile operations.\\	
Overall, the Food Technology Laboratory serves as a dynamic learning space where students can connect theoretical concepts with real-world applications, encouraging innovation in post-harvest technology, value addition, and food product development.

\begin{FeatureBox}{Computer Laboratory}
	{\textbf Computer Laboratory} at the College of Horticulture, Durgapura, situated at STR Building is equipped with 30 student systems and one instructor system, providing an excellent environment for learning the fundamentals of computer science and agricultural statistics. The lab was recently renovated. Complete wiring of the lab was upgraded as per the requirement of lab. Along with that a new central table was installed in lab and complete false-ceiling was replaced. The lab facilitates hands-on training in essential office applications such as MS Word, MS Excel, and MS PowerPoint, enabling students to develop proficiency in data handling, documentation, and presentation skills.  Additionally, the laboratory is equipped with the statistical software \textit{R Studio}, which allows students to perform data analysis, visualization, and modeling—an essential component of their coursework and research training. 
	\tcblower % --- RIGHT SIDE: IMAGE ---		
	\centering
	\includegraphics[width=\linewidth, keepaspectratio]{png/Complab.png} 
	\captionof{figure}{\small Computer Lab situated at STR, Building}		
\end{FeatureBox}		
The students of UG assemble in lab for the practical of STAT and ICT courses. Also, the lab serves as an important infrastructural asset for conducting state and national level training programs. 
\begin{FeatureBox}{Rooftop Vegetable Garden Demonstration Unit}
	{\textbf A Rooftop Vegetable Garden Demonstration Unit} was established in 2023 at the College of
	Horticulture, Durgapura, under the visionary guidance of Hon’ble Vice-Chancellor Prof. 	Balraj Singh, SKNAU, Jobner. The initiative aims to promote urban agriculture by re-purposing discarded household materials for the sustainable production of fresh and
	nutritious vegetables. Dr. L. N. Bairwa, Dean, College of Horticulture, Durgapura, this model holds immense promise for metropolitan cities such as Jaipur, where limited open space and rising demand for vegetables call for innovative solutions.
	\tcblower % --- RIGHT SIDE: IMAGE ---		
	\centering
	\includegraphics[width=\linewidth, keepaspectratio]{png/Rooftop.png} 
	\captionof{figure}{\small Dignitaries visiting the Rooftop Unit}		
\end{FeatureBox}
According to Dr. Ashok Choudhary, In -Charge, the unit has attracted wide appreciation at national and international levels. Several dignitaries have visited and praised the concept for 	its contribution to urban sustainability. Among them, Dr. R. C. Aggarwal, Former DDG (Education), ICAR, applauded it as the first model in India showcasing vegetable production through the use of discarded household materials. Erik Solheim, Norwegian diplomat and
former Executive Director of the United Nations Environment Programme, also visited the unit and appreciated its innovative approach.

\begin{FeatureBox}{Post-Harvest Technology Laboratory}
	{\textbf The Post-Harvest Technology Laboratory} at the College of Horticulture, Durgapura is a well-established facility designed to provide B.Sc. Horticulture students with practical exposure to post-harvest handling, processing, and value addition of horticultural produce. The laboratory plays a vital role in strengthening students’ understanding of post-harvest management, product development, and preservation techniques aimed at reducing losses and enhancing the shelf life and market value of fruits and vegetables.
	\tcblower % --- RIGHT SIDE: IMAGE ---		
	\centering
	\includegraphics[width=\linewidth, keepaspectratio]{png/PHTlab.png} 
	\captionof{figure}{\small Post-Harvest Technology Lab}		
\end{FeatureBox}
The lab is equipped with modern instruments and machinery such as a primary processing line, tray dryer, aloe vera gel extractor, pulper, and grinder, which enable students to learn different stages of raw material handling and product preparation. Equipment like the can sealing machine, bottle filling machine, and deep freezer help in packaging and storage studies, while analytical instruments such as the muffle furnace, centrifuge, and hot air oven support various quality and compositional analyses. Additionally, facilities like the fruit grader and autoclave aid in grading, sterilization, and hygienic processing practices.
This laboratory serves as a bridge between classroom learning and real-world applications, allowing students to develop technical skills relevant to post-harvest technology, processing, and value addition. Through hands-on training and experimentation, it encourages innovation and prepares students for professional roles in the horticulture and food processing sectors.

\paragraph{\textbf Hostels and Sports Facilities} A dedicated Girls’ Hostel is currently under construction, with the work progressing well and expected to be completed soon to provide comfortable accommodation for female students. The construction of a Boys’ Hostel has also commenced, aiming to meet the growing residential needs of the college. Additionally, a modern Sports Complex is proposed, with construction work scheduled to begin shortly, to promote physical fitness and extracurricular engagement among students.
