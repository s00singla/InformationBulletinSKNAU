{\justifying
	\chapter{HOSTEL ADMISSION AND CONDUCT RULES}
	\chapterimage{}{5em}
%	\epigraph{If I am successful in........................................................................ unquote}{--- \textup{MS Randhawa}}
	
	The Students admitted to these colleges are normally required to live in a hostel if vacancy exists unless given special permission not to be so. Campus hostels are managed by the Chief Hostel Warden with the help of Wardens for each hostel and hostel prefects in each hostel.
	
	\section*{1. ADMISSION:}
	
	\begin{enumerate}[label=1.\arabic*]
		\item Students desirous of joining the hostels shall submit applications to the Dean of the college on prescribed form which will be supplied from the office.
		\item Hostel accommodation will be mandatory for all UG, PG and PhD students admitted from academic session 2024-25. The students having critical medical issues like cardiac/T.B./Cancer/Asthmatic/Serious allergic problems will be exempted with this restriction, for which certificate will have to be produced issued by competent government medical authority.
		\item Admission to hostels will be done by the Chief Warden who will also allot the hostel and ask students to deposit the hostel fee as per rules. Allotment of seats and furniture to students will be done by the concerned Warden. The Warden will endorse a copy of allotment of hostel accommodation to the college office for cross checking of the records.
		\item On allotment of seat to a student in the hostel, the student will obtain necessary furniture and give a receipt of the articles received on a register meant for the purpose. While leaving the hostel, student should handover complete charge of the room to the Warden.
		\item The hostel fee for the READY students, who get placed at outstations like KVKs, ARS/ARSS under village attachment programme for three months, will be waived-off.
		\item No mid semester hostel admission shall be made except in case of admission to hostel for the first time. Once admitted, no student shall be allowed to leave the hostel in mid semester except when a student completes his degree programme and leaves the college. Thus, hostel fee shall be charged for the full semester except in the situations of first admission to the hostel and at completion of the programme.
	\end{enumerate}
	
	\textbf{Note:} The room/seat once allotted is final and no shifting shall be made by the students at their own level. The permission of the Warden must be obtained for any change.
	
	\section*{2. HOSTEL REGULATIONS:}
	
	\subsection*{2.1 (A) General}
	\begin{enumerate}[label=(\alph*)]
		\item The ragging of the students is strictly prohibited. The student found guilty of this will be punished severely, which may include removal from hostel and college.
		\item Students are also not allowed to carry food to the room from the kitchen/dining hall without the permission of the warden.
		\item Students should not temper with the electric fittings of the hostel, no electric appliances (Radio, Room Heater, Electric press etc.) are permitted in the rooms.
		\item Any loss or damage done to the hostel property, furniture, taps, electric fittings, utensils etc. shall be paid by the student at fault. If not traceable to any particular member, the cost of the damage will be recovered from all the members collectively.
		\item The rooms of students are liable to be checked at any time and if needed be the same may be opened even in the absence of the concerned student.
		\item Students shall not order to hostel servants and shall not interfere with their work. In case of misconduct or unsatisfactory service on the part of servants, students shall immediately report the matter to the concerned Warden.
		\item Student should not mishandle/assault hostel/mess servants.
		\item Items of common utility such as magazines, news papers, radio, television, utensils etc. should not be taken to the rooms.
		\item Absence from the hostel after 10.00 p.m. in summer and 9.00 p.m. in winter will be considered serious offence for boys. For girls, timings will be 7:30 p.m. in summer and 6:30 p.m. in winter. Concerned Warden may grant permission to stay out which shall generally be not granted for more than once a week.
		\item When any student happens to be ill, it shall be the duty of the hostel prefect/warden to report the matter to the concerned Warden.
		\item No person suffering from serious illness will be allowed (permitted) to enter into the hostel.
		\item Students shall not keep large sums or valuables in their rooms. The hostel authorities accept no responsibility for the goods lost by the students. Strict disciplinary action shall be taken against the student violating these rules or thereby creating complications for the authorities.
		\item Students shall not put up notices or convene meetings of any sort under any circumstances anywhere in the hostel compound without taking prior permission from the concerned Warden.
		\item Students should not misuse or waste electricity and water in the hostel.
		\item Each hostel will have two hostel prefects, who will be the students with the highst OGPA of that respective hostel and from the senior class. Hostel prefect will be paid remuneration of Rs. 500/- per semester along with a certificate. A student will be given the responsibility of Hostel Prefect only once during the degree programme for a maximum tenure of 2 semesters. They will perform following duties:
		\begin{enumerate}[label=(\roman*)]
			\item To see that the students that they observe hostel rules properly.
			\item To help the hostel authorities in the proper management of the hostel.
			\item To inform the Warden about any problem in the hostel in time.
			\item To inform about the absence or illness of any students in his block/wing to the Warden.
			\item The hostel prefect will oneself observe the hostel rules strictly and set an example of sense of duty, obedience, discipline and courteousness to the fellow members.
		\end{enumerate}
		\item At the time of vacation, the students are required to leave the hostel within 24 hrs. unless otherwise permitted by the Warden. Any one disobeying this rule is liable to disciplinary action.
		\item During summer vacation the students shall have to vacate the rooms. If any student wants to stay in the hostel during vacation, he/she shall have to obtain prior permission and shall have to pay the room rent and other hostel charges.
		\item Silence must be observed in the hostel at the hours when students devote to studies. Use of DJs or high volume speakers will be punishable.
		\item Students are not expected to leave station without obtaining prior sanction of the hostel authorities in writing.
		\item Throwing out waste papers, spitting, defacing walls and committing nuisance on the premises of hostels or any other kind of nuisance are punishable offences. The students are expected to maintain their rooms in a tidy condition.
		\item Intoxication in any form is an offence and the student found intoxicated shall be liable to any punishment including removal from the hostel.
		\item Keeping of weapons and intoxicants in the rooms will be treated as an offence and students will be dealt seriously including removal from the hostel.
		\item Students should not patronize peddlers, dhobies, barbers etc. unless they have permission from the Warden. Generally, no unauthorized person will be allowed to enter the hostels.
		\item In case of any problem or any quarrel in the hostel premises, the concerned students should report the matter in writing to the concerned Warden immediately (in the absence of the Warden to the Chief Warden). Direct approach to higher authorities would be considered as an act of indiscipline.
		\item Misconduct, disobedience to the hostel authorities and breach of hostel rules shall be liable to fine, suspension, removal from the hostel or college.
	\end{enumerate}
	
	\subsection*{2.1 (B)}
	All students of the college, both hostellers and day scholars using vehicles either two-wheeler or four-wheeler in the campus have to deposit/submit photocopy of papers related to their vehicles including driving license with their respective wardens in case of being hostellers and with their advisors if day scholars within a week after registration in the college positively. The concerned wardens and advisors will hand over the documents so collected by them to the Chief Hostel Warden for registering the vehicles in his office and issuing permission letter for using the vehicles in the campus. If any student is found without an authorized permission letter from the Chief Hostel warden and found in the vehicle in the college campus will be dealt with as per university rules.
	
	\subsection*{2.1 (C)}
	All hostel students have to abide by the rules framed by the university time to time.
	
	\subsection*{2.2 Visitors and Guests}
	\begin{enumerate}[label=(\alph*)]
		\item Students should take prior permission for keeping a guest in the hostel room.
		\item Visitors and guests will be required to sign in the register meant for the purpose in each hostel.
		\item A special room will be provided in the hostel for the guardians or immediate relatives visiting the wards. They will be permitted to use room for 72 hours only with a charge of Rs. 100/- per person per day. In case of girls’ hostel, only female wards will be allowed.
		\item Visitors will be allowed only between 7.00 am to 7.00 pm on working days and between 8.00 am to 10.00 am and 5.00 pm to 7.00 pm on holidays and Sundays.
	\end{enumerate}
	
	\subsection*{2.3 Punishment and use of Alcohol by the Students:}
	
	\subsubsection*{2.3.1}
	If a student is found in possession of liquor/ alcohol or in intoxicated condition in the hostel /college/university premises for the first time, he /she will be punished as under:
	\begin{enumerate}[label=(\roman*)]
		\item He/she will be issued warning
		\item He/she will be fined Rs. 1000/-
		\item He/she will have to submit an undertaking to the Dean that he/she will not repeat such activity in the future.
		\item His/her parents will be informed accordingly through his/her advisor.
	\end{enumerate}
	
	\subsubsection*{2.3.2}
	If a student is found second time in possession of liquor/ alcohol or in intoxicated condition in the hostel /college/university premises, he /she will be punished as under:
	\begin{enumerate}[label=(\roman*)]
		\item He/she will be expelled from the hostel forever
		\item He/she will be fined Rs. 2000/-
		\item He/she will be placed on conduct probation for rest of his/her study period.
		\item An undertaking will be taken by the Dean from the parents of the student
	\end{enumerate}
	
	\subsubsection*{2.3.3}
	If a student is found third time in possession of liquor/ alcohol or in intoxicated condition in the college/university premises, he /she will be expelled from the college for at least two semesters and maximum for four semesters depending upon the gravity of indiscipline.
	
	\subsubsection*{2.3.4}
	The Deans, Director Students’ Welfare of the University, Assistant Director Students’ Welfare, Chief Hostel Warden and Wardens of the constituent colleges of the University are authorized to use ‘Digital Breath Alcohol Tester’ for examining the students found in intoxicated conditions.
	
	\subsection*{2.4 Misuse of Electricity in the Hostels:}
	If any Hosteller/student is found using electric heaters/iron/room cooler (chargeable)/AC/induction/electric kettle in the hostel he/she will be punished as under:
	\begin{enumerate}[label=(\roman*)]
		\item First time a penalty of Rs. 1000/- will be imposed
		\item Second time a penalty of Rs. 2000/- will be imposed
		\item Third time if any student is found using above mentioned electric devices, Rs. 5000/ shall be imposed and will be expelled from the hostel.
	\end{enumerate}
	
}\clearpage
