{\justifying
	\chapter*{ANNEXUREs}	
	\chapterimage{}{5em}
	\vspace{-4.5cm}
	\section*{CURBING THE MENACE OF RAGGING IN HIGHER EDUCATIONAL INSTITUTIONS}
	
	It is brought to notice of the institutions, students and other various stakeholder that ragging is a criminal offence and UGC has framed regulations, on curbing the menace of ragging in higher educational institutions, in order to prohibit, prevent and eliminate the scourge of ragging. The regulations have been notified vide no. F. 1 - 16/ 2009 (CPP-II) dated 21.10.2009 and are available on UGC website \url{www.ugc.ac.in}. The abovementioned regulations are mandatory and shall apply to all universities established or incorporated by or under a central act, a Provincial act or a state / union territory and all institution recognized by or affiliated to such universities and all institutions deemed to be universities under section of (3) of the UGC Act of 1956 w.e.f. 4 July, 2009. The University will take necessary steps for its implementation, including the monitoring mechanism as per provisions provided in the above regulation and ensure its strict compliances. The following preventive measures for Anti-Ragging should also be strictly followed.
	
	\begin{enumerate}
		\item The institutions may erect suitable hoarding/bill boards/banners in prominent places within the campus to exhort the students to prevent or not to indulge in ragging and also indicating therein the names of the officials and their telephone numbers to be contacted in case of ragging.
		\item All the first-year students will file an online Anti Ragging Affidavit as per the revised UGC Guideline. For this, they may refer anti ragging section of UGC website.
		\item An Anti-Ragging Committee and squad, a dedicated cadre of wardens, and professional counselors will be operational at each institute to ensure that the directions of the Hon’ble Court of India and Justice Raghavan committee recommendations are followed without exception.
		\item The email address and contact number of the Nodal Officer of Anti-Ragging of the Universities/Colleges have been displayed on the website.
		\item The institution may also undertake other forms of campaign as it may consider appropriate for prevention of ragging.
		\item UGC has uploaded a film on anti-ragging on its website. All universities and colleges are requested to download the same and give wide publicity amongst the students, before the start of the academic session.
	\end{enumerate}
	
	Any violation of UGC regulation as cited above or if any institution fails to take adequate steps to prevent ragging or act in accordance with these regulations or fails to punish perpetrators of incidents of ragging suitably, UGC shall call for punitive action against erring institutions.  
	
	Students in distress owing to ragging related incidents can assess the toll-free helpline \textbf{1800-180-5522}.
	\newpage
%	\section*{Annexure-II}
%	% \includepdf[pages=1-2]{sow2.pdf}
%	% \begin{hindi}
%		
%		% \section*{छात्रावास नियम}
%		
%		% \begin{enumerate}
%			%     \item छात्रावासोंं के वार्डन के वर्तमान में निर्धारित कर्तव्याें का पुनरावलोकन कर इन्हंे इस प्रकार संशोधित किया जाये कि वार्डन अधिक प्रभावी ढंग से छात्रावास की सुचारु व्यवस्था कर सकें।
%			%     \item लडकों के छात्रावासों मे लडकियों/महिलाओं का तथा लडकियों के छात्रावास में लडकों/पुरुषों का प्रवेश पूर्णतया निषिद्ध हो। इस प्रावधान की कठोरता से पालना हो।
%			%     \item छात्रावास परिसर में धूम्रपान एवं मादक पदार्थो का लाना पूर्णतया निषिद्ध हो तथा मादक पदार्थो का सेवन किये हुये छात्र एवं व्यक्तियों का प्रवेश पूर्णतया वर्जित हो।
%			%     \item छात्रावास में एक विजिटर रजिस्टर संधारित हो, जिसमें छात्रावास में आने वाले प्रत्येक गैर छात्रावासी का नाम, पता एवं आने के उद्देश्य की प्रविष्टि हो।
%			%     \item बिना वार्डन की अनुमति के यदि किसी छात्रावासी के कमरे में कोई गैर छात्रावासी व्यक्ति रात्रि में पाया जाये, तो उस विद्यार्थी को तुरन्त छात्रावास से निष्कासित किया जावे।
%			%     \item छात्रावास में हथियार लाना या रखना पूर्णतया निषिद्ध हो।
%			%     \item बिना वार्डन की अनुमति के कोई भी गैर छात्रावासी व्यक्ति रात्रि में छात्रावास में निवास नहीं करेगा, यह सुनिश्चित करना वार्डन का कर्तव्य होगा।
%			%     \item गैर छात्रावासी यदि विजिटर रजिस्टर में बिना प्रविष्टि के छात्रावास में अथवा प्रविष्टि के पश्चात् भी छात्रावास के नियमों का उल्लंघन करता हुआ पाया जाये, तो उसके खिलाफ पुलिस में रिपोर्ट तुरन्त दर्ज करवाई जावे।
%			%     \item छात्रावास के परिसर में रात्रि में केवल वे ही वाहन खडे किये जा सकेंगे जो छात्रावास के हैं और जिन्होंने उसके संबंध में वार्डन को सूचना दे रखी है।
%			%     \item गैर छात्रावासी, छात्रावास परिसर में वाहन खडे नहीं कर सकेंगे। यदि ऐसा कोई वाहन पाया जावे तो वार्डन द्वारा तत्काल इसकी सूचना पुलिस को दी जावेगी।
%			%     \item यदि कोई छात्रावासी बिना अनुमति के किसी गैर छात्रावासी के वाहन को छात्रावास परिसर में खडा करता है तो उसके विरुद्ध नियमानुसार कार्यवाही की जावे।
%			%     \item वार्डन की अनुमति के बिना कोई छात्रावासी छात्र रात्रि में छात्रावास के बाहर नहीं रहेगा।
%			%     \item वार्डन को किसी भी समय आकस्मिक रूप से किसी भी कमरे को खुलवा कर निरीक्षण करने का अधिकार होगा।
%			%     \item छात्रावासियों को परिचय पत्र जारी किये जायें जो सत्र की समाप्ति पर अथवा छात्रावास खाली करने पर अथवा छात्रावास से निष्कासित किये जाने पर वापस जमा करा लिये जावें।
%			%     \item प्रत्येक शैक्षिक सत्र के अन्त में परीक्षा समाप्त होने के पश्चात छात्रावासी को अपना कमरा अनिवार्य रूप से खाली करना होगा और इसकी सूचना वार्डन को देनी होगी। यदि इस प्रकार कमरा खाली नहीं किया जाता है तो अनिवार्य रूप से कमरा खाली करवा कर कॉलेज/विश्वविद्यालय को अपने अधिकार में लेने का प्रावधान किया जावे।
%			%     \item छात्रावास में प्रवेश के विविध नियमों को इस प्रकार प्रभावी बनाया जावे कि ऐसा विद्यार्थी, जिसके खिलाफ पुलिस में रिपोर्ट दर्ज है, अथवा न्यायालय में किसी आपराधिक प्रकरण में चालान पेश हो चुका हो अथवा सजायाफ्ता हो अथवा महाविद्यालय में अनुशासनहीनता का दोषी पाया गया हों उसको छात्रावास में प्रवेश नहीं दिया जावे।
%			%     \item वार्डन के कर्तव्यों का निर्धारण इस प्रकार किया जावे कि विविध नियमों और उपरोक्त प्रस्तावित नियमों की अनुपालना सुनिश्चित कर सकें।
%			% \end{enumerate}
%		% \end{hindi}
%}