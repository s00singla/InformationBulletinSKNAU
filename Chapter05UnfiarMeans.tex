{\justifying
	\chapter{UNFAIR MEANS}
	\chapterimage{}{5em}
%	\epigraph{If I am successful in........................................................................ unquote}{--- \textup{MS Randhawa}}
	
	\section*{UNFAIR MEANS}
	
	\subsection*{1. Unfair Means shall include the following:}
	\begin{enumerate}[label=1.\arabic*]
		\item Communication or attempting to communicate with the Controller of Examination and coordinator PG Examination of the University or any person of his office or Superintendent of Examination or any person connected with the conduct of examination or with any paper setter or examiner with the object of finding out the name and address of the paper setter or examiner, for finding out the questions that have been set, in the award of marks or with the objective of unduly influencing any of them in discharge of his/her duties in connection with the examination.
		\item Giving or receiving assistance in answering the question paper to or from any other candidate/person in the examination hall or outside the examination hall.
		\item 
		\begin{enumerate}[label=(\alph*)]
			\item Having in possession during examination time any paper, book or notes that have relevance to the examination concerned.
			\item Anything written on the ink-pot cover, scales or any other instrument, or on any kind of furniture with which he/she is concerned which may have relevance to the examination concerned.
			\item Anything written or signs made on the body of the candidate, on clothes/garment, on the paper or on any substance which may have relevance to the examination concerned.
			\item Using or attempting to use any other unfair means during the examination or in connection with the examination.
			\item Smuggling in or out an answer book or impersonifying a candidate, or helping him/her in any way. 
			
			\textbf{Note:} Impersonation (false eligibility) will be considered as unfair means adopted by both the parties and would be dealt with as specified in the procedural code for dealing such cases.
			\item Copying actually from the material not to be used in the examination.
			\item Talking or whispering to other candidate or to any unauthorized person inside or outside the examination room during the examination hours without the permission of a member of the supervisory staff.
		\end{enumerate}
		\item Any other activity which may give undue advantage in the examination to any student.
	\end{enumerate}
	
	\subsection*{2. Insolent Behaviour/ Disorderly Conduct during Examinations:}
	The candidate in the examination hall or outside but within the campus of the Examination Centre during the examination shall be under the disciplinary control of the Superintendent of the Centre or his/her nominee and shall obey his/her instructions. Disorderly conduct includes:
	\begin{enumerate}[label=(\alph*)]
		\item Disobeying instructions of the Superintendent / Addl. Suptd / Asstt. Suptd/ Invigilator or any member of the Flying Squad.
		\item Threatening, intimidating or assaulting the Suptd, Asstt. Suptd, Invigilator, any member of the Flying Squad or any other member of staff working at the examination before, during or after the examination hours.
		\item Misbehaving with the Suptd, Asstt. Suptd, Invigilator(s), any member of the flying squad or any other member working at the examination centre in connection with the examination before, during or after the examination hours.
		\item Leaving the examination room before expiry of half an hour after the commencement of the examination, or leaving the examination room without obtaining the permission of the invigilator or without handing over the answer book to the invigilator or without signing the attendance sheet.
		\item Tearing of or mutilating an answer-book (Main or Supplementary) or any part thereof.
		\item Disturbing or disrupting the conduct of examination or attempting to do so.
		\item Insisting or compelling any other candidate to leave the examination room or to disturb/boycott the examination.
		\item Bringing into the Examination Hall/Centre any weapon or any other material objected to by the Invigilator/Centre Supdt. or any other member of the Supervisory staff.
		\item Appearing in the examination without being in possession of the admission card unless permitted by the Centre Superintendent.
		\item Refusing to be searched by the Invigilator/Centre Superintendent/any other member of the Supervisory staff/any member of the Flying Squad, or obstructing or hindering such search in the examination hall/verandah, urinal etc.
	\end{enumerate}
	
	\subsection*{3. Norms of Punishment to Candidates Guilty of Unfair Means and/or Disorderly Conduct:}
	\begin{enumerate}[label=3.\arabic*]
		\item If a candidate is found guilty of seeking admission to an examination by making a false representation pertaining to his/her eligibility to appear at the examination he/she shall be disqualified from appearing at any examination for a period of two to four years including the present examination.
		\item The Rajasthan Public Examination (Prevention of Unfair means) Act, 1992 and Rajasthan Public Examination (Prevention of Unfairmeans) (Amendment) Act, 2022 will be applicable for all the examinations conducted by the University/College and the Examination Superintendents are empowered to take suitable action as per provision of the act in matter of unfair means.
		\item Where a candidate is found having in his/her possession or within his/her reach any material relevant to the syllabus of the examination paper concerned but has not copied from or used it:
		\begin{enumerate}[label=(\alph*)]
			\item If the behaviour of the candidate on being caught is satisfactory: Present examination shall be cancelled provided that if the material found in possession of the candidate is of insignificant nature, the punishment may be relaxed to the extent of cancellation of the examination of that particular paper (theory or practical as the case may be) and he/she will be treated as having obtained ``Zero'' mark in that paper with all the consequences to follow.
			\item If the behaviour of the candidate on being caught is unsatisfactory: Present examination shall be cancelled and he/she shall be further debarred for one subsequent main examination if the examination is held once a year, or two subsequent semesters if the examination is held twice a year.
		\end{enumerate}
		\textbf{Note:} If a candidate uses resistance or violence against the invigilator or any other person on examination duty, the punishment may be enhanced according to the gravity of offence.
		
		\item Where a candidate is found to have copied from or used the material caught:
		\begin{enumerate}[label=(\alph*)]
			\item If the behaviour of the candidate on being caught is satisfactory: Present examination shall be cancelled and he/she shall be further debarred for one subsequent annual examination or two subsequent semester examination. If the material found in possession of the candidate and/or the extent of copying done by the candidate is of insignificant nature, the punishment may be relaxed to the extent of cancelling the present examination only.
			\item If the behaviour of the candidate on being caught is unsatisfactory: Present examination shall be cancelled and he/she shall be further debarred from appearing at two subsequent examinations if held once a year or debarred from four subsequent examinations, if the examination is held twice a year.
		\end{enumerate}
		\textbf{Note:} 
		\begin{enumerate}[label=(\roman*)]
			\item If the candidate uses resistance or violence against the invigilator or any person on examination duty or consistently refuses to obey the instructions of the Superintendent, the above punishment may be enhanced according to the gravity of offence.
			\item The phrase ``present examination is cancelled'' refers to cancellation of only theory papers and practicals (whenever held). However, if a candidate has offered dissertation, viva-voce/field work in lieu of any paper, the same will not be cancelled in case the whole examination is cancelled.
		\end{enumerate}
		\item If a candidate is found guilty of having torn or mutilated an answer book (main or supplementary) or any part thereof, his/her present examination shall be cancelled and he/she shall be further debarred for one subsequent main examination, if the examination is held once a year, or two subsequent semesters if the examination is held twice a year.
		
		\item If a candidate is found guilty of swallowing or attempting to swallow a note or paper or running away with it or throwing it away with the intention of destroying evidence, his/her present examination shall be cancelled and he/she shall be further debarred for one subsequent main examination, if the examination is held once a year, or two subsequent semesters if the examination is held twice a year.
		
		\item If a candidate is found guilty of smuggling in an answer book or taking out an answer book, his/her present examination shall be cancelled and he/she shall be further debarred for three subsequent main examinations if held once a year or six subsequent semester examinations if held twice a year.
		
		\item If a candidate is found guilty of insertion of a paper in an answer book or of getting an answer book written through another person, his/her present examination shall be cancelled and he/she shall be further debarred for three subsequent main examinations if held once a year or six subsequent semester examinations if held twice a year.
		
		\item If a candidate is found guilty of allowing somebody else to impersonate him/her or if he/she impersonates somebody else, his/her present examination shall be cancelled and he/she shall be further debarred for four subsequent main examinations if held once a year or eight subsequent semester examinations if held twice a year. Both the impersonator and the candidate shall be dealt with in the same manner.
		
		\item If a candidate is found guilty of communicating or attempting to communicate with an examiner with the objective of influencing him in the award of marks, his/her present examination shall be cancelled and he/she shall be further debarred for one subsequent main examination if held once a year or two subsequent semester examinations if held twice a year.
		
		\item If a candidate is found guilty of threatening, assaulting, or using abusive language against any member of the supervisory staff or any person on examination duty, his/her present examination shall be cancelled and he/she shall be further debarred for four subsequent main examinations if held once a year or eight subsequent semester examinations if held twice a year. The case shall also be reported to the police for legal action.
		
		\item If a candidate is found guilty of approaching directly or indirectly any member of the University authority/examination staff with an offer of illegal gratification or inducement in any form, his/her present examination shall be cancelled and he/she shall be further debarred for four subsequent main examinations if held once a year or eight subsequent semester examinations if held twice a year. The matter shall also be reported to the police for legal action.
		
		\item If a candidate is found guilty of communicating with other candidate(s) inside or outside the examination hall with the object of giving or receiving help in answering the questions, his/her present examination shall be cancelled and he/she shall be further debarred for one subsequent main examination if held once a year or two subsequent semester examinations if held twice a year.
		
		\item If a candidate is found guilty of disobeying the instructions of the Superintendent/Invigilator/any member of the Flying Squad, his/her present examination shall be cancelled and he/she shall be further debarred for one subsequent main examination if held once a year or two subsequent semester examinations if held twice a year.
		
		\item If a candidate is found guilty of leaving the examination hall without handing over his/her answer book to the invigilator concerned or without signing the attendance sheet, his/her present examination shall be cancelled and he/she shall be further debarred for one subsequent main examination if held once a year or two subsequent semester examinations if held twice a year.
		
		\item If a candidate is found guilty of disturbing or disrupting the conduct of examination or attempting to do so, his/her present examination shall be cancelled and he/she shall be further debarred for two subsequent main examinations if held once a year or four subsequent semester examinations if held twice a year.
		
		\item If a candidate is found guilty of bringing into the examination hall any weapon or any objected material, his/her present examination shall be cancelled and he/she shall be further debarred for four subsequent main examinations if held once a year or eight subsequent semester examinations if held twice a year. The case shall also be reported to the police for legal action.
		
	\end{enumerate}
	
	\noindent
	\textbf{Additional Provision:} In addition to the provisions laid down to deal with the cases of unfair means during the examination by the University, such candidates will also be dealt with additionally in pursuance of the Rajasthan Public Examination (Prevention of Unfair means) Act, 1992 and Rajasthan Public Examination (Prevention of Unfairmeans) (Amendment) Act, 2022. The salient features related to University Examination system and other aspects are summarized below.
	\newpage
	{
		\setlist[enumerate,1]{label=\arabic*.}
		\setlist[itemize]{left=1.2cm}
		
		\begin{center}
			\Large \textbf{The Rajasthan Public Examination (Prevention of Unfairmeans) Act, 1992}\\[4pt]
			\normalsize (Act No. 27 of 1992)\\
			(Received the assent of the Governor on the 8th day of November, 1992)
		\end{center}
		
		\noindent
		\textbf{An Act} to prevent the leakage of question papers and use of unfairmeans at public examinations and to provide for matters connected therewith and incidental thereto. 
		
		\medskip
		\noindent
		Be it enacted by the Rajasthan State Legislature in the Forty-third Year of the Republic of India as follows:—
		
		\section*{1. Short title, extent and commencement}
		\begin{enumerate}[label=(\arabic*)]
			\item This Act may be called the Rajasthan Public Examination (Prevention of Unfairmeans) Act, 1992.
			\item It shall extend to the whole of the State of Rajasthan.
			\item It shall come into force at once.
		\end{enumerate}
		
		\section*{2. Definitions}
		In this Act—
		\begin{enumerate}[label=(\alph*)]
			\item ``examination centre'' means any place fixed for holding public examination and includes the entire premises attached thereto;
			\item ``public examination'' means any of the examination specified in the Schedule;
			\item ``unfairmeans'' in relation to an examination while answering question in a public examination, means the unauthorised help from any person, or from any material written, recorded or printed, in any form whatsoever or the use of any unauthorised telephonic, wireless or electronic or other instrument or gadget; and
			\item the words and expressions used herein and not defined, but defined in the Indian Penal Code (45 of 1860), have the meanings respectively assigned to them in that Code.
		\end{enumerate}
		
		\section*{3. Prohibition of use of unfairmeans}
		No person shall use unfairmeans at any public examination.
		
		\section*{4. Unauthorised possession or disclosure of question paper}
		No person who is not lawfully authorised or permitted by virtue of his duties so to do shall, before the time fixed for distribution of question papers to examinees at a public examination—
		\begin{enumerate}[label=(\alph*)]
			\item procure or attempt to procure or possess, such question paper or any portion or copy thereof; or
			\item impart or offer to impart, information which he knows or has reason to believe to be related to, or derived from or to have a bearing upon such question paper.
		\end{enumerate}
		
		\section*{5. Prevention of leakage by person entrusted with examination work}
		No person who is entrusted with any work pertaining to public examination shall, except where he is permitted by virtue of his duties so to do, directly or indirectly divulge or cause to be divulged or make known to any other person any information or part thereof which has come to his knowledge by virtue of the work being so entrusted to him.
		
		\section*{6. Penalty}
		Whoever contravenes or attempts to contravene or abets the contravention of the provisions of section 3 or section 4 or section 5, shall be punished with imprisonment for a term which may extend to three years or with fine which may extend to two thousand rupees or with both.
		
		\section*{7. Penalty for offence with preparation to cause hurt}
		Whoever commits an offence punishable under section 6 having made preparation for causing death of any person or causing hurt to any person or assaulting any person or for wrongfully restraining any person or for wrongful restraint shall be punished with imprisonment for a term which may extend to three years and shall also be liable to fine which may extend to five thousand rupees.
		
		\section*{8. Power to amend Schedule}
		The State Government may, by notification in the Official Gazette, include in the Schedule any other public examination in respect of which it considers necessary to apply the provisions of this Act and upon the publication in the Official Gazette, the Schedule shall be deemed to have been amended accordingly.
		
		\section*{The Schedule (Section 2)}
		{%\small
		\begin{enumerate}[label=\arabic*.]
			\item Any examination conducted by the Board of Secondary Education for Rajasthan under the Rajasthan Secondary Examination Act, 1957 (Act No. 42 of 1957).
			\item Any examination conducted by any University established by law in India.
			\item Any examination conducted by the Rajasthan Public Service Commission or Union Public Service Commission.
			\item Any examination conducted by any Autonomous College in the State.\footnote{Inserted vide Gazette Notification dated 15th April 1993 (published 16th April 1993).}
			\item Any examination conducted by Railway Recruitment Board, Ajmer.\footnote{Inserted vide Gazette Notification dated 18th Nov 1999 (published 20th Nov 1999).}
			\item Any examination conducted by the Rajasthan High Court.\footnote{Inserted vide Gazette Notification dated 6th June 2014.}
			\item Any examination conducted by the Rajasthan Subordinate and Ministerial Service Selection Board.\footnote{Inserted vide Gazette Notification dated 13th May 2015 (published 18th May 2015).}
			\item Any examination of All India Trade Test (AITT) conducted by the Directorate General of Training (DGT), New Delhi, under the Ministry of Skill Development and Entrepreneurship, Government of India.\footnote{Inserted vide Gazette Notification dated 15th Jan 2019 (published 21st Jan 2019).}
		\end{enumerate}
	}
		
		% \begin{center}
			% \textbf{\large THE RAJASTHAN PUBLIC EXAMINATION (MEASURES FOR PREVENTION OF UNFAIR MEANS IN RECRUITMENT) ACT, 2022}\\
			% (Act No. 6 of 2022)\\
			% (Authorised English Translation)\\[1em]
			% \textit{(Received the assent of the Governor on the 5th day of April, 2022)}
			% \end{center}
		
		% \noindent
		% \textbf{An Act} to provide for effective measures to prevent and curb the offences of leakage of question papers and use of unfair means at public examinations for the purpose of recruitment to any post under the State Government including autonomous bodies, authorities, boards or corporations, and to provide for designated courts for the trial of such offences and for matters connected therewith or incidental thereto.  
		
		% \medskip
		
		% \noindent
		% \textbf{Be it enacted by the Rajasthan State Legislature in the Seventy-third Year of the Republic of India, as follows:}
		
		% \section*{1. Short title, extent and commencement}
		% \begin{enumerate}[label=(\arabic*)]
			%     \item This Act may be called the Rajasthan Public Examination (Measures for Prevention of Unfair Means in Recruitment) Act, 2022.
			%     \item It shall extend to the whole of the State of Rajasthan.
			%     \item It shall come into force on such date, as the State Government may, by notification in the Official Gazette, appoint.
			% \end{enumerate}
		
		% \section*{2. Definitions}
		% In this Act, unless the subject or context otherwise requires,—
		% \begin{enumerate}[label=(\alph*)]
			%     \item ``conduct of public examination'' means and includes preparation, printing, supervision, coding, processing, storing, transportation, distribution and collection of question papers, answer sheets, OMR sheets and result sheets, evaluation, declaration of result, etc.;
			%     \item ``examination authority'' means an examination authority as specified in the Schedule-I;
			%     \item ``examination center'' means any institution or part thereof or any other place fixed and used for the holding of a public examination and includes the entire premises attached thereto;
			%     \item ``examinee'' means a person who has been granted permission by the concerning authority to appear in a public examination, and includes a person authorized to act as scribe on his behalf;
			%     \item ``public examination'' means examination for the purpose of recruitment to any post under the State Government including autonomous bodies, authorities, boards or corporations as specified in the Schedule-II;
			%     \item ``unfair means'' includes—
			%     \begin{enumerate}[label=(\roman*)]
				%         \item in relation to an examinee, to take unauthorized help in public examination from any person or group directly or indirectly or from any material written, recorded, copied or printed, in any form whatsoever, or use of any unauthorized electronic or mechanical instrument or gadget;
				%         \item in relation to any person—
				%         \begin{enumerate}[label=\Roman*.]
					%             \item to impersonate or leak or attempt to leak or conspire to leak question paper; or
					%             \item to procure or attempt to procure or possess or attempt to possess question paper in unauthorized manner; or
					%             \item to solve or attempt to solve or seek assistance to solve question paper in unauthorized manner; and
					%             \item directly or indirectly assist the examinee in the public examination in unauthorized manner.
					%         \end{enumerate}
				%     \end{enumerate}
			%     \textit{Explanation.— Any person also includes an examinee.}
			%     \item the words and expressions used herein and not defined, but defined in the Indian Penal Code, 1860 (Central Act No. 45 of 1860), shall have the same meanings respectively assigned to them in that code.
			% \end{enumerate}
		
		% \section*{3. Prohibition of use of unfair means}
		% No person shall use unfair means at any public examination.
		
		% \section*{4. Possession and disclosure of question paper}
		% No person authorized by virtue of his duties in conduct of public examination shall, before the time fixed for opening and distribution of question papers—
		% \begin{enumerate}[label=(\alph*)]
			%     \item open, leak or procure or attempt to procure, possess or solve such question paper or any portion or a copy thereof; or
			%     \item give any confidential information or promise to give such confidential information to any person or examinee, where such confidential information is related to or in reference to such question paper.
			% \end{enumerate}
		
		% \section*{5. Prevention of leakage by person entrusted or engaged with examination work}
		% No person, who is entrusted or engaged with any work pertaining to public examination shall, except where he is permitted by virtue of his duties so to do, directly or indirectly divulge or cause to be divulged or make known to any other person any information or part thereof which has come to his knowledge by virtue of the work being so entrusted to him.
		
		% \section*{6. Unauthorized possession or disclosure of question paper and answer sheet or OMR sheet in any form}
		% No person who is not lawfully authorized or permitted by virtue of his duties to do so, shall, before the time fixed for the distribution of question papers—
		% \begin{enumerate}[label=(\alph*)]
			%     \item procure or attempt to procure or possess, such question paper or answer sheet or OMR sheet or any portion or copy thereof in any form; or
			%     \item impart or offer to impart, such information which he knows or has reason to believe to be related to, or derived from or to have a bearing upon such question paper.
			% \end{enumerate}
		
		% \section*{7. Prohibition to enter in examination center}
		% No person who is not entrusted or engaged with the work pertaining to public examination or conduct of public examination or who is not an examinee, shall enter the premises of the examination center.
		
		% \section*{8. No place other than examination center shall be used for public examination}
		% No person who is entrusted or engaged with the work pertaining to public examination shall use or cause to be used any place, other than the examination center, for the purpose of holding public examination.
		
		% \section*{9. Offences by Management, Institution or others}
		% \begin{enumerate}[label=(\arabic*)]
			%     \item Whenever an offence under this Act has been committed by Management or Institution or Limited Liability Partnership or others, every person who at the time the offence was committed was in charge of, or was responsible to the Management or Institution or Limited Liability Partnership or others for conduct of the business of the Management or Institution or Limited Liability Partnership or others, as well as the Management or Institution or Limited Liability Partnership or others, shall be deemed to be guilty of the offence and shall be liable to be proceeded against and punished accordingly:
			
			%     \medskip
			%     \textit{Provided that nothing contained in this sub-section shall render any such person liable to any punishment under this Act if he proves that the offence was committed without his knowledge and that he exercised all due diligence to prevent the commission of such offence.}
			
			%     \item Notwithstanding anything contained in sub-section (1) where an offence under this Act has been committed by Management or Institution or Limited Liability Partnership or others and it is proved that the offence has been committed with the consent or connivance of, or is attributable to any neglect on the part of, any director, partner, manager, secretary or other officer of the Management or Institution or Limited Liability Partnership or others, such director, partner, manager, secretary or other officer shall also be deemed to be guilty of the offence and shall be liable to be proceeded against and punished accordingly.
			% \end{enumerate}
	}
	
}\clearpage
